% !TEX root = ../main.tex
\chapter*{BẢNG PHÂN CÔNG NHIỆM VỤ}
\addcontentsline{toc}{chapter}{BẢNG PHÂN CÔNG NHIỆM VỤ}

\begin{table}[H]
  \centering
  \renewcommand{\arraystretch}{1.5}
  \setlength{\tabcolsep}{5pt}
  \begin{tabular}{|c|p{4.5cm}|c|p{5cm}|c|}
    \hline
    \textbf{STT} & \textbf{Họ và tên} & \textbf{MSSV} & \textbf{Nhiệm vụ được giao} & \textbf{Đánh giá} \\
    \hline

    % Thành viên 1
    1 & Nguyễn Trường Sơn & 20227148 &
    \begin{itemize}[nosep, leftmargin=1em, before=\vspace{-0.5\baselineskip}]
      \item Nhóm trưởng (Leader)
      \item Quản lý tiến độ dự án
      \item Thiết kế Biểu đồ ca (usecase), Biểu đồ luồng, Biểu đồ ERD
      \item Tổng hợp báo cáo
    \end{itemize} & A+ \\
    \hline

    % Thành viên 2
    2 & Trần Bảo Châu & 20216799 & Thiết kế sơ đồ phân tích lớp & B \\
    \hline

    % Thành viên 3
    3 & Đinh Minh Hà & 20227194 &
    \begin{itemize}[nosep, leftmargin=1em, before=\vspace{-0.5\baselineskip}]
      \item Thiết kế giao diện, Phân tích các biểu đồ ca quản lý
      \item Vẽ sơ đồ chuyển màn hình
    \end{itemize} & A \\
    \hline

    % Thành viên 4
    4 & Nguyễn Doanh Thái & 20237483 &
    \begin{itemize}[nosep, leftmargin=1em, before=\vspace{-0.5\baselineskip}]
      \item Thiết kế giao diện, Phân tích các biểu đồ ca quản lý
      \item Vẽ sơ đồ chuyển màn hình
    \end{itemize} & A \\
    \hline

    % Thành viên 5
    5 & Cao Phạm Minh Tuấn & 20237492 & Phân tích các biểu đồ ca nghiệp vụ bán hàng (POS) & A \\
    \hline

    % Thành viên 6
    6 & Phạm Xuân Vỹ & 20237496 &
    Phân tích các biểu đồ ca nghiệp vụ kho & A \\
    \hline
  \end{tabular}
  \caption*{Bảng tổng hợp phân công công việc}
\end{table}

\vspace{1cm}
\textbf{Cam kết của nhóm:} Các thành viên đã hoàn thành đầy đủ nhiệm vụ được giao đúng thời hạn và đảm bảo chất lượng công việc.

\newpage

\chapter{LỜI NÓI ĐẦU}

Mô hình cửa hàng tiện lợi (CVS - Convenience Store) như Circle K, GS25 và 7-11 rất phổ biến và được ưa chuộng với nhóm học sinh, sinh viên bởi đặc tính tiện lợi, mở 24/7 và cung cấp chỗ ngồi. Không chỉ bán đồ tạp hoá, CVS còn cung cấp cả thực phẩm chế biến như mì trộn, bánh bao hoặc đồ uống như Milo đá, Coca tươi. Bởi vậy nên bên cạnh các nghiệp vụ quản lý truyền thống của một cửa hàng như POS hay quản lý nhân sự, CVS còn bao gồm các nghiệp vụ đặc biệt của một cửa hàng đồ ăn nhanh, ví dụ như chế biến hay quản lý hạn sử dụng trong ngày thay vì quản lý theo ngày/tháng.

Trong báo cáo này, chúng em sẽ trình bày chi tiết các bước thực hiện dự án, từ phân tích yêu cầu, thiết kế hệ thống, đến triển khai và kiểm thử Hệ thống Quản lý CVS. Các kỹ thuật và công nghệ đã được áp dụng để đảm bảo hệ thống hoạt động chính xác, ổn định và có thể truy vết. Đặc biệt, Hệ thống quản lý CVS được xây dựng để có thể hoạt động liên tục trong thời gian dài, hỗ trợ nhân viên phục vụ hàng trăm đơn hàng trong giờ cao điểm mà không gặp phải sự cố hay gây khó khăn cho nhân viên sử dụng. 

Chúng em hy vọng rằng, với Hệ thống Quản lý CVS, các chủ cửa hàng CVS sẽ có thêm lựa chọn đơn giản mà mạnh mẽ để quản lý một mô hình rất mới, rất độc đáo và tiềm năng, cạnh tranh với các chuỗi CVS lớn. Với dự án này, chúng em mong rằng sẽ góp phần thúc đẩy doanh số ngành bán lẻ. 

Chúng em xin chân thành cảm ơn sự hỗ trợ và hướng dẫn từ thầy Lê Hải Hà, cùng sự hợp tác của các thành viên trong nhóm, để dự án có thể hoàn thành đúng tiến độ và đạt được những mục tiêu đề ra.
