% !TEX root = ../main.tex
\chapter{THIẾT KẾ CHƯƠNG TRÌNH}

\section{Thiết kế cơ sở dữ liệu}

\subsection{Tổng quan mô hình dữ liệu}

Hệ thống Quản lý Cửa hàng Tiện lợi (CVS) được thiết kế với 32 bảng dữ liệu, phân chia thành 8 miền nghiệp vụ chính. Mô hình dữ liệu được xây dựng theo các nguyên tắc:

\begin{itemize}
  \item \textbf{Chuẩn hóa dữ liệu:} Các bảng được thiết kế ở dạng chuẩn 3NF để tránh dư thừa và đảm bảo tính nhất quán.
  \item \textbf{Khả năng mở rộng đa cửa hàng:} Tất cả các bảng có phạm vi theo cửa hàng đều chứa khóa ngoại \texttt{store\_id}, sẵn sàng cho việc mở rộng hệ thống.
  \item \textbf{Truy vết đầy đủ:} Mọi thay đổi tồn kho đều được ghi nhận trong bảng \texttt{stock\_movements} với đầy đủ thông tin: ai thực hiện, thời điểm, lý do và lô bị ảnh hưởng.
  \item \textbf{Quản lý theo lô:} Hàng hóa được quản lý theo từng lô nhập với hạn sử dụng riêng biệt, hỗ trợ cơ chế FIFO (First-In-First-Out).
\end{itemize}

\subsection{Các miền dữ liệu}

Hệ thống được chia thành 8 miền dữ liệu chính, mỗi miền phục vụ một nhóm ca sử dụng cụ thể:

\begin{longtable}{|c|p{4cm}|p{6cm}}
  \hline
  \textbf{STT} & \textbf{Miền dữ liệu}  & \textbf{Các bảng}                                                                                                                                                                                              \\
  \hline
  \endfirsthead

  \hline
  \textbf{STT} & \textbf{Miền dữ liệu}  & \textbf{Các bảng}                                                                                                                                                                                              \\
  \hline
  \endhead

  \hline
  \endfoot

  \hline
  \caption{Phân chia các miền dữ liệu trong hệ thống} \label{tab:data-domains}                                                                                                                                                                           \\
  \endlastfoot

  1            & Xác thực \& Người dùng & \texttt{staff}, \texttt{sessions}                                                                                                                                                                              \\
  \hline
  2            & Quản lý ca \& Tiền mặt & \texttt{shifts}, \texttt{cash\_drops}                                                                                                                                                                          \\
  \hline
  3            & Nhân sự                & \texttt{shift\_templates}, \texttt{work\_schedules}, \texttt{timekeeping}, \texttt{leave\_requests}, \texttt{payroll\_periods}, \texttt{payroll\_records}                                                      \\
  \hline
  4            & Danh mục sản phẩm      & \texttt{categories}, \texttt{physical\_items}, \texttt{sales\_items}, \texttt{recipes}, \texttt{recipe\_items}, \texttt{combos}, \texttt{combo\_items}                                                         \\
  \hline
  5            & Quản lý kho            & \texttt{suppliers}, \texttt{receivings}, \texttt{receiving\_items}, \texttt{inventory\_batches}, \texttt{stock\_movements}, \texttt{conversions}, \texttt{inventory\_counts}, \texttt{inventory\_count\_items} \\
  \hline
  6            & Bán hàng               & \texttt{orders}, \texttt{order\_items}, \texttt{order\_item\_batches}, \texttt{payments}, \texttt{receipts}, \texttt{remakes}                                                                                  \\
  \hline
  7            & Tiêu hủy \& Kiểm soát  & \texttt{waste\_logs}                                                                                                                                                                                           \\
  \hline
  8            & Cấu hình \& Khuyến mại & \texttt{stores}, \texttt{payment\_method\_configs}, \texttt{printer\_configs}, \texttt{alerts}, \texttt{discount\_rules}, \texttt{discount\_rule\_items}                                                       \\
  \hline
\end{longtable}

\subsection{Mô tả chi tiết các bảng dữ liệu}

\subsubsection{Miền Xác thực và Người dùng}

\textbf{Bảng staff} - Lưu trữ thông tin nhân viên:

\begin{longtable}{|p{3.5cm}|p{3cm}|p{7cm}|}
  \hline
  \textbf{Thuộc tính} & \textbf{Kiểu dữ liệu} & \textbf{Mô tả}                        \\
  \hline
  \endfirsthead
  \endhead
  \endfoot
  \hline
  \caption{Cấu trúc bảng staff}                                                       \\
  \endlastfoot

  id                  & SERIAL                & Khóa chính, tự động tăng              \\
  \hline
  name                & VARCHAR(100)          & Họ tên nhân viên                      \\
  \hline
  email               & VARCHAR(150)          & Email đăng ký (dùng để khôi phục PIN) \\
  \hline
  pin\_hash           & VARCHAR(255)          & Mã PIN đã được băm                    \\
  \hline
  role                & VARCHAR(20)           & Vai trò: STAFF hoặc MANAGER           \\
  \hline
  store\_id           & INTEGER               & Khóa ngoại đến bảng stores            \\
  \hline
  hourly\_rate        & INTEGER               & Lương theo giờ (đơn vị: đồng)         \\
  \hline
  is\_active          & BOOLEAN               & Trạng thái hoạt động                  \\
  \hline
  created\_at         & TIMESTAMPTZ           & Thời điểm tạo                         \\
  \hline
  updated\_at         & TIMESTAMPTZ           & Thời điểm cập nhật cuối               \\
  \hline
\end{longtable}

\textbf{Bảng sessions} - Theo dõi phiên đăng nhập:

\begin{longtable}{|p{3.5cm}|p{3cm}|p{7cm}|}
  \hline
  \textbf{Thuộc tính} & \textbf{Kiểu dữ liệu} & \textbf{Mô tả}                                    \\
  \hline
  \endfirsthead
  \endhead
  \endfoot
  \hline
  \caption{Cấu trúc bảng sessions}                                                                \\
  \endlastfoot

  id                  & SERIAL                & Khóa chính                                        \\
  \hline
  staff\_id           & INTEGER               & Khóa ngoại đến staff                              \\
  \hline
  shift\_id           & INTEGER               & Khóa ngoại đến shifts                             \\
  \hline
  login\_at           & TIMESTAMPTZ           & Thời điểm đăng nhập                               \\
  \hline
  logout\_at          & TIMESTAMPTZ           & Thời điểm đăng xuất (được phép trống)             \\
  \hline
  logout\_reason      & VARCHAR(20)           & Lý do đăng xuất: MANUAL, AUTO\_LOCK, FAST\_SWITCH \\
  \hline
\end{longtable}

\subsubsection{Miền Quản lý ca và Tiền mặt}

\textbf{Bảng shifts} - Quản lý ca làm việc:

\begin{longtable}{|p{3.5cm}|p{3cm}|p{7cm}|}
  \hline
  \textbf{Thuộc tính} & \textbf{Kiểu dữ liệu} & \textbf{Mô tả}                      \\
  \hline
  \endfirsthead
  \endhead
  \endfoot
  \hline
  \caption{Cấu trúc bảng shifts}                                                    \\
  \endlastfoot

  id                  & SERIAL                & Khóa chính                          \\
  \hline
  staff\_id           & INTEGER               & Nhân viên mở ca                     \\
  \hline
  store\_id           & INTEGER               & Cửa hàng                            \\
  \hline
  start\_time         & TIMESTAMPTZ           & Thời điểm mở ca                     \\
  \hline
  end\_time           & TIMESTAMPTZ           & Thời điểm đóng ca                   \\
  \hline
  opening\_cash       & INTEGER               & Tiền mặt đầu ca                     \\
  \hline
  closing\_cash       & INTEGER               & Tiền mặt cuối ca (thực tế đếm)      \\
  \hline
  expected\_cash      & INTEGER               & Tiền mặt dự kiến (tính từ hệ thống) \\
  \hline
  cash\_variance      & INTEGER               & Chênh lệch (thực tế - dự kiến)      \\
  \hline
  variance\_reason    & VARCHAR(255)          & Lý do chênh lệch                    \\
  \hline
  status              & VARCHAR(20)           & Trạng thái: OPEN hoặc CLOSED        \\
  \hline
\end{longtable}

\subsubsection{Miền Nhân sự}

\textbf{Bảng timekeeping} - Chấm công nhân viên:

\begin{longtable}{|p{3.5cm}|p{3cm}|p{7cm}|}
  \hline
  \textbf{Thuộc tính} & \textbf{Kiểu dữ liệu} & \textbf{Mô tả}                            \\
  \hline
  \endfirsthead
  \endhead
  \endfoot
  \hline
  \caption{Cấu trúc bảng timekeeping}                                                     \\
  \endlastfoot

  id                  & SERIAL                & Khóa chính                                \\
  \hline
  staff\_id           & INTEGER               & Nhân viên chấm công                       \\
  \hline
  work\_date          & DATE                  & Ngày làm việc                             \\
  \hline
  clock\_in           & TIMESTAMPTZ           & Thời điểm vào làm                         \\
  \hline
  clock\_out          & TIMESTAMPTZ           & Thời điểm ra về                           \\
  \hline
  total\_hours        & DECIMAL(5,2)          & Tổng giờ làm                              \\
  \hline
  status              & VARCHAR(20)           & PENDING, CLOCKED\_IN, COMPLETED, ADJUSTED \\
  \hline
  adjusted\_by        & INTEGER               & Người điều chỉnh (nếu có)                 \\
  \hline
  adjustment\_reason  & VARCHAR(255)          & Lý do điều chỉnh                          \\
  \hline
\end{longtable}

\textbf{Bảng leave\_requests} - Đơn xin nghỉ phép:

\begin{longtable}{|p{3.5cm}|p{3cm}|p{7cm}|}
  \hline
  \textbf{Thuộc tính} & \textbf{Kiểu dữ liệu} & \textbf{Mô tả}                           \\
  \hline
  \endfirsthead
  \endhead
  \endfoot
  \hline
  \caption{Cấu trúc bảng leave\_requests}                                                \\
  \endlastfoot

  id                  & SERIAL                & Khóa chính                               \\
  \hline
  staff\_id           & INTEGER               & Nhân viên xin nghỉ                       \\
  \hline
  leave\_type         & VARCHAR(30)           & Loại nghỉ: ANNUAL, SICK, PERSONAL, OTHER \\
  \hline
  start\_date         & DATE                  & Ngày bắt đầu nghỉ                        \\
  \hline
  end\_date           & DATE                  & Ngày kết thúc nghỉ                       \\
  \hline
  reason              & TEXT                  & Lý do xin nghỉ                           \\
  \hline
  status              & VARCHAR(20)           & PENDING, APPROVED, REJECTED              \\
  \hline
  reviewed\_by        & INTEGER               & Quản lý duyệt đơn                        \\
  \hline
  reviewer\_notes     & VARCHAR(255)          & Ghi chú của quản lý                      \\
  \hline
\end{longtable}

\textbf{Bảng work\_schedules} - Đăng ký lịch làm việc:

\begin{longtable}{|p{3.5cm}|p{3cm}|p{7cm}|}
  \hline
  \textbf{Thuộc tính} & \textbf{Kiểu dữ liệu} & \textbf{Mô tả}                         \\
  \hline
  \endfirsthead
  \endhead
  \endfoot
  \hline
  \caption{Cấu trúc bảng work\_schedules}                                              \\
  \endlastfoot

  id                  & SERIAL                & Khóa chính                             \\
  \hline
  staff\_id           & INTEGER               & Nhân viên đăng ký                      \\
  \hline
  shift\_template\_id & INTEGER               & Ca làm việc mẫu                        \\
  \hline
  work\_date          & DATE                  & Ngày làm việc                          \\
  \hline
  status              & VARCHAR(20)           & PENDING, APPROVED, REJECTED, CANCELLED \\
  \hline
  approved\_by        & INTEGER               & Quản lý phê duyệt                      \\
  \hline
\end{longtable}

\subsubsection{Miền Danh mục sản phẩm}

\textbf{Bảng physical\_items} - Hàng hóa vật lý (SKU trong kho):

\begin{longtable}{|p{3.5cm}|p{3cm}|p{7cm}|}
  \hline
  \textbf{Thuộc tính} & \textbf{Kiểu dữ liệu} & \textbf{Mô tả}                       \\
  \hline
  \endfirsthead
  \endhead
  \endfoot
  \hline
  \caption{Cấu trúc bảng physical\_items}                                            \\
  \endlastfoot

  id                  & SERIAL                & Khóa chính                           \\
  \hline
  name                & VARCHAR(150)          & Tên sản phẩm vật lý                  \\
  \hline
  barcode             & VARCHAR(50)           & Mã vạch (tùy chọn)                   \\
  \hline
  storage\_unit       & VARCHAR(30)           & Đơn vị lưu kho (gói, thùng, vỉ)      \\
  \hline
  prepared\_unit      & VARCHAR(30)           & Đơn vị sau chế biến (miếng, quả)     \\
  \hline
  conversion\_ratio   & DECIMAL(10,4)         & Tỷ lệ chuyển đổi (1 gói = 10 miếng)  \\
  \hline
  expiry\_hours       & INTEGER               & Thời gian hết hạn sau chế biến (giờ) \\
  \hline
  reorder\_threshold  & INTEGER               & Ngưỡng cảnh báo bổ sung              \\
  \hline
\end{longtable}

\textbf{Bảng sales\_items} - Sản phẩm bán (SKU bán hàng):

\begin{longtable}{|p{3.5cm}|p{3cm}|p{7cm}|}
  \hline
  \textbf{Thuộc tính}   & \textbf{Kiểu dữ liệu} & \textbf{Mô tả}                               \\
  \hline
  \endfirsthead
  \endhead
  \endfoot
  \hline
  \caption{Cấu trúc bảng sales\_items}                                                         \\
  \endlastfoot

  id                    & SERIAL                & Khóa chính                                   \\
  \hline
  category\_id          & INTEGER               & Danh mục sản phẩm                            \\
  \hline
  physical\_item\_id    & INTEGER               & Liên kết đến hàng vật lý (nếu bán trực tiếp) \\
  \hline
  name                  & VARCHAR(150)          & Tên sản phẩm bán                             \\
  \hline
  price                 & INTEGER               & Giá bán (đồng)                               \\
  \hline
  near\_expiry\_price   & INTEGER               & Giá khi cận date                             \\
  \hline
  vat\_rate             & DECIMAL(5,2)          & Thuế suất VAT (0, 5, 8, 10\%)                \\
  \hline
  requires\_preparation & BOOLEAN               & Có cần chế biến không                        \\
  \hline
\end{longtable}

\textbf{Bảng recipes và recipe\_items} - Công thức chế biến:

Bảng \texttt{recipes} lưu thông tin công thức, liên kết 1-1 với \texttt{sales\_items} có \texttt{requires\_preparation = true}. Bảng \texttt{recipe\_items} lưu chi tiết nguyên liệu với số lượng cần thiết. Đây là cơ sở cho cơ chế \textbf{Backflushing} - tự động trừ kho nguyên liệu khi bán món chế biến.

\subsubsection{Miền Quản lý kho}

\textbf{Bảng inventory\_batches} - Quản lý lô hàng:

\begin{longtable}{|p{3.5cm}|p{3cm}|p{7cm}|}
  \hline
  \textbf{Thuộc tính} & \textbf{Kiểu dữ liệu} & \textbf{Mô tả}                       \\
  \hline
  \endfirsthead
  \endhead
  \endfoot
  \hline
  \caption{Cấu trúc bảng inventory\_batches}                                         \\
  \endlastfoot

  id                  & SERIAL                & Khóa chính                           \\
  \hline
  physical\_item\_id  & INTEGER               & Sản phẩm vật lý                      \\
  \hline
  receiving\_item\_id & INTEGER               & Nguồn từ phiếu nhập (nếu có)         \\
  \hline
  source\_batch\_id   & INTEGER               & Lô nguồn (nếu là kết quả chuyển đổi) \\
  \hline
  quantity\_initial   & INTEGER               & Số lượng ban đầu                     \\
  \hline
  quantity\_remaining & INTEGER               & Số lượng còn lại                     \\
  \hline
  created\_at         & TIMESTAMPTZ           & Thời điểm tạo lô                     \\
  \hline
  expires\_at         & TIMESTAMPTZ           & Thời điểm hết hạn                    \\
  \hline
  batch\_state        & VARCHAR(30)           & STORAGE hoặc PREPARED                \\
  \hline
\end{longtable}

Cơ chế FIFO được thực hiện bằng cách sắp xếp các lô theo \texttt{created\_at ASC} và trừ từ lô cũ nhất trước.

\textbf{Bảng stock\_movements} - Nhật ký biến động kho:

Đây là bảng trung tâm cho việc truy vết, ghi nhận mọi thay đổi tồn kho với các loại:
\begin{itemize}
  \item \texttt{RECEIVING}: Nhập hàng từ nhà cung cấp
  \item \texttt{SALE}: Bán hàng (trừ kho)
  \item \texttt{WASTE}: Tiêu hủy
  \item \texttt{EXPIRY}: Hết hạn tự động
  \item \texttt{CONVERSION\_IN/OUT}: Chuyển đổi trạng thái
  \item \texttt{ADJUSTMENT}: Điều chỉnh kiểm kê
  \item \texttt{VOID}: Hủy đơn hàng (hoàn kho)
\end{itemize}

\subsubsection{Miền Bán hàng}

\textbf{Bảng orders} - Đơn hàng:

\begin{longtable}{|p{3.5cm}|p{3cm}|p{7cm}|}
  \hline
  \textbf{Thuộc tính}   & \textbf{Kiểu dữ liệu} & \textbf{Mô tả}                                   \\
  \hline
  \endfirsthead
  \endhead
  \endfoot
  \hline
  \caption{Cấu trúc bảng orders}                                                                   \\
  \endlastfoot

  id                    & SERIAL                & Khóa chính                                       \\
  \hline
  staff\_id             & INTEGER               & Nhân viên tạo đơn                                \\
  \hline
  shift\_id             & INTEGER               & Ca làm việc                                      \\
  \hline
  order\_number         & VARCHAR(10)           & Số đơn hàng (4 chữ số, reset hàng ngày)          \\
  \hline
  subtotal              & INTEGER               & Tổng tiền hàng                                   \\
  \hline
  discount\_total       & INTEGER               & Tổng giảm giá                                    \\
  \hline
  vat\_total            & INTEGER               & Tổng VAT                                         \\
  \hline
  grand\_total          & INTEGER               & Tổng thanh toán                                  \\
  \hline
  status                & VARCHAR(30)           & DRAFT, PENDING\_PAYMENT, PAID, COMPLETED, VOIDED \\
  \hline
  remake\_of\_order\_id & INTEGER               & Đơn gốc (nếu là đơn làm lại)                     \\
  \hline
\end{longtable}

\textbf{Bảng order\_item\_batches} - Theo dõi FIFO khi bán:

Bảng này ghi nhận chi tiết việc trừ kho từ những lô nào cho mỗi dòng đơn hàng, đảm bảo khả năng truy vết và hỗ trợ hoàn kho chính xác khi hủy đơn.

\subsection{Biểu đồ quan hệ thực thể (ERD)}

\begin{landscape}
  \begin{figure}[p]
    \centering
    \includegraphics[width=\linewidth,height=\textheight,keepaspectratio]{assets/diagram-erd.png}
    \caption{Biểu đồ ERD tổng quan hệ thống CVS}
    \label{fig:erd-overview}
  \end{figure}
\end{landscape}

\subsection{Các ràng buộc toàn vẹn}

\subsubsection{Ràng buộc khóa chính và khóa ngoại}

Tất cả các bảng sử dụng khóa chính tự động tăng (\texttt{SERIAL}). Các quan hệ giữa các bảng được thực thi thông qua ràng buộc khóa ngoại với \texttt{ON DELETE RESTRICT} để ngăn chặn việc xóa dữ liệu đang được tham chiếu.

\subsubsection{Ràng buộc nghiệp vụ}

\begin{itemize}
  \item \textbf{FIFO bắt buộc:} Mọi thao tác trừ kho phải tuân thủ FIFO, không có ngoại lệ.
  \item \textbf{Không bán hàng hết hạn:} Hệ thống chặn việc bán từ các lô có \texttt{expires\_at < NOW()}.
  \item \textbf{Không sửa đơn hàng:} Đơn hàng sau khi thanh toán không thể sửa, chỉ có thể hủy và tạo lại.
  \item \textbf{Tiền lưu dạng số nguyên:} Tất cả giá trị tiền tệ lưu dưới dạng INTEGER (đơn vị: đồng) để tránh sai số làm tròn.
  \item \textbf{PIN được băm:} Mã PIN không lưu plaintext, chỉ lưu giá trị đã băm.
  \item \textbf{Soft delete cho audit:} Các bản ghi quan trọng (đơn hàng, phiếu tiêu hủy) không được xóa cứng, chỉ đánh dấu trạng thái.
\end{itemize}

\subsection{Quy ước đặt tên}

\begin{longtable}{|p{4cm}|p{4cm}|p{5cm}|}
  \hline
  \textbf{Đối tượng} & \textbf{Quy ước}                & \textbf{Ví dụ}                                  \\
  \hline
  \endfirsthead
  \endhead
  \endfoot
  \hline
  \caption{Quy ước đặt tên trong CSDL}                                                                   \\
  \endlastfoot

  Tên bảng           & snake\_case, số nhiều           & \texttt{order\_items}, \texttt{physical\_items} \\
  \hline
  Tên cột            & snake\_case                     & \texttt{created\_at}, \texttt{unit\_price}      \\
  \hline
  Khóa chính         & \texttt{id}                     & \texttt{id SERIAL PRIMARY KEY}                  \\
  \hline
  Khóa ngoại         & \texttt{\{bảng\}\_id}           & \texttt{staff\_id}, \texttt{order\_id}          \\
  \hline
  Chỉ mục            & \texttt{idx\_\{bảng\}\_\{cột\}} & \texttt{idx\_orders\_created\_at}               \\
  \hline
  Timestamp          & TIMESTAMPTZ                     & Lưu kèm timezone                                \\
  \hline
  Tiền tệ            & INTEGER                         & Đơn vị: đồng Việt Nam                           \\
  \hline
\end{longtable}

\section{Thiết kế giao diện người dùng}

\subsection{Chuẩn hoá cấu hình màn hình}

\subsubsection{Quy định về hiển thị}
\begin{itemize}
  \item \textbf{Độ phân giải:}
    \begin{itemize}
      \item Màn hình POS (Bán hàng): Tối ưu cho độ phân giải 1024x768 pixels (tỷ lệ 4:3) hoặc 1280x800 pixels, hỗ trợ thao tác cảm ứng với các nút bấm lớn.
      \item Màn hình Quản lý: Tối ưu cho độ phân giải 1366x768 pixels trở lên trên trình duyệt web máy tính.
    \end{itemize}
  \item \textbf{Thông báo:}
    \begin{itemize}
      \item \textbf{Thông báo lỗi/cảnh báo:} Hiển thị dạng Popup (Modal) ở chính giữa màn hình, làm mờ nền phía sau, yêu cầu người dùng xác nhận hoặc sửa lỗi trước khi tiếp tục.
      \item \textbf{Thông báo thành công:} Hiển thị dạng Toast message ở góc trên bên phải, tự động ẩn sau 3 giây.
    \end{itemize}
  \item \textbf{Định dạng dữ liệu:}
    \begin{itemize}
      \item \textbf{Số tiền:} Sử dụng dấu chấm (.) để phân cách hàng nghìn (Ví dụ: 100.000 đ). Đơn vị tiền tệ mặc định là VNĐ.
      \item \textbf{Ngày tháng:} Định dạng dd/MM/yyyy (Ví dụ: 30/01/2026).
      \item \textbf{Ký tự cho phép:} Chuỗi ký tự bao gồm chữ cái, chữ số, dấu phẩy, dấu chấm, dấu cách, dấu gạch dưới và gạch nối.
    \end{itemize}
\end{itemize}

\subsubsection{Cơ chế điều khiển}
\begin{itemize}
  \item \textbf{Kiểm tra dữ liệu đầu vào:}
    \begin{itemize}
      \item \textbf{Kiểm tra trường bắt buộc:} Kiểm tra các trường bắt buộc ngay khi người dùng rời khỏi ô nhập liệu (làm mờ) hoặc khi nhấn lưu.
      \item \textbf{Kiểm tra định dạng:} Kiểm tra định dạng đúng của Email, Số điện thoại, Mã vạch.
      \item \textbf{Logic nghiệp vụ:} Ví dụ: Số lượng tồn kho không được âm, Tiền khách đưa $\ge$ Tổng tiền thanh toán.
    \end{itemize}
  \item \textbf{Dịch chuyển màn hình:}
    \begin{itemize}
      \item Thiết kế theo hướng phẳng, hạn chế các khung chồng chéo.
      \item Các màn hình chức năng (Bán hàng, Kho, Nhân sự) được tách biệt rõ ràng thông qua thanh điều hướng.
      \item Các tác vụ phụ (như Hướng dẫn sử dụng, Chi tiết nhanh) có thể hiển thị dạng Popup đè lên màn hình chính, làm mờ màn hình dưới để tập trung sự chú ý.
    \end{itemize}
\end{itemize}

\subsubsection{Luồng màn hình hệ thống}

Trình tự xuất hiện các màn hình trong kịch bản sử dụng thông thường:

\begin{enumerate}
  \item \textbf{Khởi động:} Màn hình chờ $\rightarrow$ Màn hình Đăng nhập.
  \item \textbf{Trang chủ (Sau đăng nhập):}
    \begin{itemize}
      \item Với Nhân viên: Chuyển thẳng vào Màn hình POS (Bán hàng).
      \item Với Quản lý: Chuyển vào Dashboard (Bảng điều khiển thống kê).
    \end{itemize}
  \item \textbf{Các luồng chức năng chính:}
    \begin{itemize}
      \item \textbf{Bán hàng:} POS $\rightarrow$ Chọn món $\rightarrow$ Chỉnh sửa đơn (Topping/Ghi chú) $\rightarrow$ Thanh toán (Tiền mặt/QR) $\rightarrow$ In hóa đơn $\rightarrow$ Kết thúc.
      \item \textbf{Quản lý kho:} Danh sách nhập/xuất $\rightarrow$ Tạo phiếu nhập/xuất $\rightarrow$ Chọn hàng hóa $\rightarrow$ Xác nhận $\rightarrow$ Cập nhật tồn kho.
      \item \textbf{Nhân sự:} Lịch làm việc $\rightarrow$ Đăng ký ca $\rightarrow$ Xin nghỉ phép $\rightarrow$ Chấm công.
    \end{itemize}
  \item \textbf{Màn hình lỗi (Error):} Khi hệ thống gặp sự cố, một thông điệp rõ ràng sẽ hiện lên thông báo vấn đề và hướng dẫn cách khắc phục (thử lại hoặc liên hệ kỹ thuật).
\end{enumerate}

\subsection{Các ảnh màn hình}

% =============================================
% NHÓM 1: ĐĂNG NHẬP & NHÂN VIÊN BÁN HÀNG
% =============================================

\begin{figure}[H]
  \centering
  \includegraphics[width=0.9\textwidth]{assets/anh_man_hinh/dang_nhap.png}
  \caption{Màn hình đăng nhập}
  \label{fig:dang_nhap}
\end{figure}

\begin{figure}[H]
  \centering
  \includegraphics[width=0.9\textwidth]{assets/anh_man_hinh/nhan_vien_ban_hang.png}
  \caption{Màn hình nhân viên bán hàng}
  \label{fig:nhan_vien_ban_hang}
\end{figure}

\begin{figure}[H]
  \centering
  \includegraphics[width=0.9\textwidth]{assets/anh_man_hinh/lich_lam_viec.png}
  \caption{Màn hình nhân viên bán hàng - Lịch làm việc}
  \label{fig:lich_lam_viec}
\end{figure}

\begin{figure}[H]
  \centering
  \includegraphics[width=0.9\textwidth]{assets/anh_man_hinh/dang_ky_ca_lam.png}
  \caption{Lịch làm việc - Đăng ký ca làm}
  \label{fig:dang_ky_ca_lam}
\end{figure}

\begin{figure}[H]
  \centering
  \includegraphics[width=0.9\textwidth]{assets/anh_man_hinh/xin_nghi.png}
  \caption{Đơn xin nghỉ phép}
  \label{fig:xin_nghi}
\end{figure}

% =============================================
% NHÓM 2: POS, THANH TOÁN & BẾP
% =============================================

\begin{figure}[H]
  \centering
  \includegraphics[width=0.9\textwidth]{assets/anh_man_hinh/pos.png}
  \caption{Màn hình POS}
  \label{fig:pos}
\end{figure}

\begin{figure}[H]
  \centering
  \includegraphics[width=0.9\textwidth]{assets/anh_man_hinh/che_bien.png}
  \caption{Màn hình POS - Quản lý thực phẩm chế biến}
  \label{fig:che_bien}
\end{figure}

\begin{figure}[H]
  \centering
  \includegraphics[width=0.9\textwidth]{assets/anh_man_hinh/QR.png}
  \caption{Màn hình thanh toán của khách hàng (Quét QR)}
  \label{fig:qr}
\end{figure}

\begin{figure}[H]
  \centering
  \includegraphics[width=0.9\textwidth]{assets/anh_man_hinh/hoan_tien_lam_lai.png}
  \caption{Màn hình Hoàn tiền / Làm lại món}
  \label{fig:hoan_tien_lam_lai}
\end{figure}

\begin{figure}[H]
  \centering
  \includegraphics[width=0.9\textwidth]{assets/anh_man_hinh/quan_ly_nhap_xuat.png}
  \caption{Quản lý nhập xuất kho}
  \label{fig:nhap_xuat}
\end{figure}

\begin{figure}[H]
  \centering
  \includegraphics[width=0.9\textwidth]{assets/anh_man_hinh/popup_ket_thuc_ca.png}
  \caption{Popup kết thúc ca làm việc}
  \label{fig:popup_ket_thuc_ca}
\end{figure}

% =============================================
% NHÓM 3: QUẢN LÝ (DASHBOARD & THIẾT LẬP)
% =============================================

\begin{figure}[H]
  \centering
  \includegraphics[width=0.9\textwidth]{assets/anh_man_hinh/man_hinh_quan_ly.png}
  \caption{Giao diện quản lý chính}
  \label{fig:man_hinh_quan_ly}
\end{figure}

\begin{figure}[H]
  \centering
  \includegraphics[width=0.9\textwidth]{assets/anh_man_hinh/dashboard_tong_quan.png}
  \caption{Hệ thống quản lý - Dashboard tổng quan}
  \label{fig:dashboard}
\end{figure}

\begin{figure}[H]
  \centering
  \includegraphics[width=0.9\textwidth]{assets/anh_man_hinh/cong_thuc_che_bien.png}
  \caption{Quản lý công thức chế biến}
  \label{fig:cong_thuc}
\end{figure}

\begin{figure}[H]
  \centering
  \includegraphics[width=0.9\textwidth]{assets/anh_man_hinh/quan_ly_combo.png}
  \caption{Quản lý Combo}
  \label{fig:combo}
\end{figure}

\begin{figure}[H]
  \centering
  \includegraphics[width=0.9\textwidth]{assets/anh_man_hinh/quan_ly_khuyen_mai.png}
  \caption{Quản lý khuyến mại}
  \label{fig:khuyen_mai}
\end{figure}

% =============================================
% NHÓM 4: KHO & DANH MỤC HÀNG HÓA
% =============================================

\begin{figure}[H]
  \centering
  \includegraphics[width=0.9\textwidth]{assets/anh_man_hinh/quan_ly_danh_muc_nhap.png}
  \caption{Quản lý danh mục hàng - Hàng nhập}
  \label{fig:dm_nhap}
\end{figure}

\begin{figure}[H]
  \centering
  \includegraphics[width=0.9\textwidth]{assets/anh_man_hinh/quan_ly_danh_muc_ban.png}
  \caption{Quản lý danh mục hàng - Hàng bán}
  \label{fig:dm_ban}
\end{figure}


\begin{figure}[H]
  \centering
  \includegraphics[width=0.9\textwidth]{assets/anh_man_hinh/quan_ly_nhap_xuat_kho.png}
  \caption{Chi tiết danh sách nhập xuất kho}
  \label{fig:nhap_xuat_kho}
\end{figure}

\begin{figure}[H]
  \centering
  \includegraphics[width=0.9\textwidth]{assets/anh_man_hinh/phieu_nhap_kho.png}
  \caption{Phiếu nhập kho}
  \label{fig:phieu_nhap_kho}
\end{figure}

\begin{figure}[H]
  \centering
  \includegraphics[width=0.9\textwidth]{assets/anh_man_hinh/phieu_huy_hang.png}
  \caption{Phiếu hủy hàng}
  \label{fig:phieu_huy_hang}
\end{figure}

\begin{figure}[H]
  \centering
  \includegraphics[width=0.9\textwidth]{assets/anh_man_hinh/phieu_kiem_ke.png}
  \caption{Phiếu kiểm kê kho}
  \label{fig:phieu_kiem_ke}
\end{figure}

% =============================================
% NHÓM 5: NHÂN SỰ & HỆ THỐNG
% =============================================

\begin{figure}[H]
  \centering
  \includegraphics[width=0.9\textwidth]{assets/anh_man_hinh/danh_sach_nhan_vien.png}
  \caption{Quản lý danh sách nhân viên}
  \label{fig:ql_nhan_vien}
\end{figure}

\begin{figure}[H]
  \centering
  \includegraphics[width=0.9\textwidth]{assets/anh_man_hinh/phan_ca_lam_viec.png}
  \caption{Phân ca làm việc}
  \label{fig:phan_ca}
\end{figure}

\begin{figure}[H]
  \centering
  \includegraphics[width=0.9\textwidth]{assets/anh_man_hinh/quan_ly_tai_khoan.png}
  \caption{Quản lý tài khoản}
  \label{fig:ql_tai_khoan}
\end{figure}

\begin{figure}[H]
  \centering
  \includegraphics[width=0.9\textwidth]{assets/anh_man_hinh/cau_hinh_he_thong.png}
  \caption{Cấu hình hệ thống}
  \label{fig:cau_hinh}
\end{figure}

\begin{figure}[H]
  \centering
  \includegraphics[width=0.9\textwidth]{assets/anh_man_hinh/bao_cao_thong_ke.png}
  \caption{Báo cáo thống kê}
  \label{fig:bao_cao}
\end{figure}

\subsection{Sơ đồ chuyển màn hình}
\begin{figure}[H]
  \centering
  \includegraphics[width=0.9\textwidth]{assets/anh_man_hinh/chuyen_man_hinh.png}
  \caption{Sơ đồ chuyển màn hình}
  \label{fig:chuyen_man_hinh}
\end{figure}
% Chèn sơ đồ nếu có
% \includegraphics...