% !TEX program = lualatex 
% =================== CÀI ĐẶT CHUNG ===================
\documentclass[a4paper, twoside]{report}
%\documentclass[a4paper, 13pt, twoside]{extreport} % Bằng một cách thần kỳ nào đấy, thuật toán co giãn của lệnh này lại nhỏ hơn lệnh changefontsizes
% (*) 'a4paper': Quy định khổ giấy A4.
% (*) '13pt': Cỡ chữ cơ bản cho toàn văn bản là 13pt.
% (*) 'twoside': Định dạng tài liệu để in hai mặt, lề trái và lề phải của trang chẵn và lẻ sẽ khác nhau (inside/outside margins).
\usepackage{scrextend} % Cung cấp thêm các lệnh mở rộng cho lớp tài liệu chuẩn.
\changefontsizes{13pt} % Lệnh từ gói scrextend, áp dụng cỡ font chữ là 13 cho toàn văn bản.

% =================== GÓI LỆNH BỐ CỤC & TRANG ===================
\usepackage[top=2cm, bottom=2cm, left=3.5cm, right=2.5cm]{geometry} %Thiết lập lề trang (trên, dưới, trái, phải).
\usepackage{fancyhdr} % Tùy biến header và footer
\fancypagestyle{plain}{%
  \fancyhf{} % Xóa toàn bộ nội dung, định dạng của header và footer.
  \fancyfoot[RO,RE]{\thepage} % (*) Đặt số trang ở góc phải cho cả trang lẻ (Right Odd) và trang chẵn (Right Even).
  \renewcommand{\headrulewidth}{0pt} % Điều chỉnh độ dày của đương kẻ ngang header, ở đây mục đích là bỏ đường kẻ ngang ở header (Độ dày 0pt)
\renewcommand{\footrulewidth}{0pt}} % Điều chỉnh độ dày của đương kẻ ngang footer, ở đây mục đích là bỏ đường kẻ ngang ở footer (Độ dày 0pt)
% Lệnh trên để tùy chỉnh lại định dạng trang 'plain' (thường dùng cho trang đầu chương).
\setlength{\headheight}{16.3pt} % Đặt chiều cao cho vùng header để tránh lỗi cảnh báo của fancyhdr.

\usepackage{pdflscape} % Hỗ trợ trang ngang, giúp đưa các bảng có kích thước đặt theo chiều ngang giấy
\usepackage{setspace} % Hỗ trợ giãn dòng
\setstretch{1.1} % Giãn dòng 1.1
%\onehalfspacing % Giãn dòng 1.5, theo đồ án SoICT là giãn 1.5, còn theo file quy chế chính thức ctt là giãn 1.1, sinh viên tùy chỉnh nếu muốn rộng hơn.
\usepackage{afterpage} % Hỗ trợ can thiệp ngắt trang, ví dụ để chèn một trang trống sau trang hiện tại.

% =================== GÓI LỆNH HÌNH ẢNH, BẢNG, CHÚ THÍCH ===================
\usepackage{graphicx} %Cho phép chèn hình ảnh với lệnh \includegraphics.
\graphicspath{{figures/}{../figures/}} % Thư mục chứa các hình ảnh, tùy chọn đường dẫn dựa vào người dùng
\usepackage{array} % Tạo bảng, mở rộng khả năng tùy biến các cột trong môi trường bảng (tabular).
\usepackage{multirow} % Hợp nhất các ô trong bảng.
\usepackage{subcaption} % Hỗ trợ tạo các hình ảnh/bảng con trong cùng một môi trường figure/table.
\usepackage{float} % Cung cấp tùy chọn [H] để ``ghim'' đối tượng (hình, bảng) tại đúng vị trí khai báo.
\usepackage[font=small,labelfont=bf]{caption} % Tùy chỉnh định dạng của chú thích (caption): 'font=small' cho chữ nhỏ hơn, 'labelfont=bf' in đậm nhãn (ví dụ: ``Hình 1.1'').
\usepackage{capt-of} % Cho phép tạo chú thích (caption) cho các đối tượng nằm ngoài môi trường float (như figure, table).
\captionsetup[table]{position=top} % Đảm bảo chú thích của bảng luôn nằm ở phía trên bảng.
\counterwithin{figure}{chapter} % Đánh số hình ảnh theo số chương. Ví dụ: Hình 1.1, 1.2,..
\counterwithin{table}{chapter}  % Đánh số bảng biểu theo số chương. Ví dụ: Bảng 1.1, 1.2,...

% =================== GÓI LỆNH TOÁN HỌC & THUẬT TOÁN ===================
\usepackage{amsmath} % Hỗ trợ nhiều môi trường và lệnh để soạn thảo công thức toán học phức tạp.
\usepackage{amssymb} % Bổ sung thêm kí hiệu về toán học (ví dụ: \mathbb).
\usepackage{amsthm} % Cho phép tạo môi trường định lý, ví dụ
\theoremstyle{definition} %Định dạng: Tiêu đề in đậm, nội dung in đứng, dành cho các phần mang tính giải thích, không phải chứng minh, như Định nghĩa (Definition), Điều kiện (Condition), và Ví dụ (Example). Có thể tùy chỉnh thành \theoremstyle{plain}, \theoremstyle{remark}. Tuy nhiên sử dụng trong quy cách đồ án sử dụng definition.
\newtheorem{example}{Ví dụ}[chapter] % Tạo môi trường ``Ví dụ'', được đánh số theo chương.
\counterwithin{equation}{chapter} % Đánh số phương trình theo chương
\renewcommand{\theequation}{PT \thechapter.\arabic{equation}} % Đổi định dạng số phương trình (ví dụ: ``PT 1.1'').
\usepackage[ruled,vlined]{algorithm2e} % Hỗ trợ viết các giải thuật, mã giả (pseudocode).

% =================== GÓI LỆNH NGÔN NGỮ & FONT ===================
\usepackage{fontspec}
\setmainfont{Times New Roman}

\usepackage[vietnamese]{babel}
\usepackage{indentfirst} % Thụt đầu dòng ở dòng đầu tiên trong đoạn

% =================== GÓI LỆNH MỤC LỤC & TIÊU ĐỀ ===================
\usepackage{titlesec} % Giúp tùy chỉnh sâu định dạng của các tiêu đề (chương, mục,...).
\usepackage{titletoc} % Cung cấp các lệnh để tùy chỉnh định dạng của bảng mục lục (Table of Contents).
\setcounter{secnumdepth}{3} % Cho phép đánh số subsubsection trong báo cáo (report).
\setcounter{tocdepth}{3} % Cho phép chèn subsubsection vào bảng mục lục

% =================== GÓI LỆNH TÀI LIỆU THAM KHẢO ===================
\usepackage{csquotes} % Recommended for biblatex
\DeclareQuoteAlias{english}{vietnamese} % Alias Vietnamese to English quotes to fix ``No style for language'' warning
\usepackage[backend=biber, style=ieee]{biblatex} % Quản lý và trích dẫn tài liệu tham khảo. 'backend=biber' là engine xử lý mạnh mẽ, 'style=ieee' là định dạng trích dẫn kiểu số [1], [2].
\DeclareLanguageMapping{vietnamese}{english} % Map Vietnamese to English for biblatex as Vietnamese is not natively supported
\addbibresource{references.bib} % Liên kết tới file .bib chứa danh sách tài liệu tham khảo.
\DefineBibliographyStrings{english}{
  bibliography = {TÀI LIỆU THAM KHẢO},
} % Việt hóa tiêu đề của danh mục tài liệu tham khảo.
\usepackage{appendix} % Hỗ trợ tạo và quản lý phần Phụ lục.
\renewcommand\appendixname{PHỤ LỤC} % Việt hóa tiêu đề trang bắt đầu phụ lục.
\renewcommand\appendixpagename{PHỤ LỤC} % Việt hóa tên chương phụ lục trong header.
\renewcommand\appendixtocname{PHỤ LỤC} % Việt hóa tên chương phụ lục trong mục lục.

% =================== GÓI CHO NỘI DUNG CHUYÊN DỤNG (CODE, GLOSSARY,...) ===================
\usepackage{listings} % Dùng để chèn các đoạn mã nguồn (source code) vào tài liệu.
% \include{lstlisting} % Gọi file lstlisting.tex chứa các thiết lập định dạng riêng cho gói listings.
\usepackage{xurl} % Hỗ trợ ngắt dòng các đường link URL dài một cách thông minh, tránh tràn lề.
\usepackage[nonumberlist, nopostdot, nogroupskip, acronym]{glossaries}  % Tạo danh sách các từ viết tắt và thuật ngữ.
\usepackage{glossary-superragged} % Style bổ sung cho gói glossaries.
\setglossarystyle{superraggedheaderborder} % Thiết lập style hiển thị cho danh sách thuật ngữ.
\usepackage{fancybox} % Để tạo các loại khung hộp trang trí cho văn bản.

% !TEX root = main.tex
%\makeglossaries
\makenoidxglossaries

% Danh mục thuật ngữ và từ viết tắt
\newglossaryentry{Use case}{
	type=\acronymtype,
 name={Use cases},
	description={Biểu đồ ca sử dụng},
	first={Use cases}
}
\newglossaryentry{API}{
	type=\acronymtype,
	name={API},
	description={Giao diện lập trình ứng dụng (Application Programming Interface)},
	first={API}
}
\newglossaryentry{CVS}{
	type=\acronymtype,
 name={CVS},
	description={Cửa hàng tiện lợi (Convenience Store)},
	first={CVS}
}
% !TEX root = main.tex

% Packages required for Use Case command
\usepackage[table]{xcolor}
\usepackage{longtable}
\usepackage{float}

% Define template colors
\definecolor{ucLabel}{HTML}{C5E0B4} % Light Green
\definecolor{ucHeader}{HTML}{F8CBAD} % Peach/Orange

% Use Case Environment using longtable for page breaking and fixed width
\newcounter{ucstepcnt}
\newcounter{ucaltcnt}

% Column widths
% Col 1: 2.5cm (Centered)
% Col 2: 12.0cm
% Total content: 14.5cm

\newenvironment{usecasetable}[2]{%
	\setcounter{ucstepcnt}{0}%
	\setcounter{ucaltcnt}{0}%
	\renewcommand{\arraystretch}{1.4}%
	\begin{longtable}{|>{\columncolor{ucLabel}\bfseries\centering\arraybackslash}p{2.5cm}|p{12.0cm}|}
		\hline
		Mã ca  & #1 \\ \hline
		Tên ca & #2 \\ \hline
		}{%
	\end{longtable}
}

% Pre-condition
\newcommand{\ucpre}[1]{%
\rowcolor{ucLabel}
\multicolumn{2}{|c|}{\bfseries Tiền điều kiện} \\ \hline
\multicolumn{2}{|p{15cm}|}{#1} \\ \hline
}

% Main flow header
\newcommand{\ucmain}{%
	\rowcolor{ucLabel}
	\multicolumn{2}{|c|}{\bfseries Luồng sự kiện chính} \\ \hline
	\rowcolor{ucHeader}
	\bfseries STT & \centering\bfseries\arraybackslash Thực hiện bởi / Hành động \\ \hline
}

% Main flow step
\newcommand{\ucstep}[3]{%
	\stepcounter{ucstepcnt}%
	\theucstepcnt & \textbf{#2}: #3 \\ \hline
}

% Alternative flow header
\newcommand{\ucalthead}{%
	\rowcolor{ucLabel}
	\multicolumn{2}{|c|}{\bfseries Luồng sự kiện thay thế} \\ \hline
	\rowcolor{ucHeader}
	\bfseries STT & \centering\bfseries\arraybackslash Thực hiện bởi / Hành động \\ \hline
}

% Alternative flow step
\newcommand{\ucalt}[3]{%
	\stepcounter{ucaltcnt}%
	\theucaltcnt & \textbf{#2}: #3 \\ \hline
}

% Post condition
\newcommand{\ucpost}[1]{%
\rowcolor{ucLabel}
\multicolumn{2}{|c|}{\bfseries Hậu điều kiện} \\ \hline
\multicolumn{2}{|p{15cm}|}{#1} \\ \hline
}

% =================== GÓI LỆNH TIỆN ÍCH KH_C ===================
\usepackage{unicode-math}
\usepackage{microtype}
\usepackage{enumitem} % Tùy chỉnh sâu các danh sách (itemize, enumerate).
\usepackage{hyperref} % Tạo các liên kết siêu văn bản trong tài liệu PDF (ví dụ: từ mục lục tới chương, từ trích dẫn tới tài liệu tham khảo).

% =================== ĐỊNH NGHĨA TIÊU ĐỀ DỰ ÁN ===================
\newcommand{\projecttitle}{Hệ thống quản lý Cửa hàng tiện lợi}
\title{\bf ĐỒ ÁN \MakeUppercase{\projecttitle}} % Khai báo tiêu đề cho tài liệu.
\author{NGUYỄN TRƯỜNG SƠN} % Khai báo tác giả cho tài liệu.
\hypersetup{
    pdfborder={0 0 0},
    pdftitle={\projecttitle},
    pdfauthor={Nguyễn Trường Sơn}
} % Tùy chỉnh hyperref: 'pdfborder' xóa khung viền quanh các link.
\usepackage[all]{hypcap} % Cho phép các tham chiếu siêu liên kết trỏ đến đầu của hình ảnh/bảng biểu, thay vì trỏ đến chú thích (caption) ở dưới.

% ========== ĐỊNH NGHĨA LẠI CÁC TIÊU ĐỀ (Chương, Mục) ==========
\titleformat{\chapter}[hang]{\centering\bfseries}{CHƯƠNG \thechapter. }{0pt}{}
\titlespacing*{\chapter}{0pt}{-20pt}{20pt}

\titleformat{\section}[hang]{\bfseries}{\thesection}{1em}{}
\titlespacing*{\section}{0pt}{1.5ex plus 1ex minus .2ex}{1.5ex plus .2ex}

\titleformat{\subsection}[hang]{\bfseries}{\thesubsection}{1em}{}
\titlespacing*{\subsection}{0pt}{1.5ex plus 1ex minus .2ex}{1.5ex plus .2ex}

\titleformat{\subsubsection}[hang]{\itshape}{\thesubsubsection}{1em}{}
\titlespacing*{\subsubsection}{0pt}{1.5ex plus 1ex minus .2ex}{1.5ex plus .2ex}

% ========== KHOẢNG CÁCH ĐOẠN & CÁC THIẾT LẬP KHÁC ==========
\setlength{\parskip}{6pt} %Khoảng cách xuống dòng theo đồ án SoICT gốc
%\setlength{\parskip}{3pt} %Khoảng cách xuống dòng theo file đồ án chính thức của đại học (Spacing Before): 3pt, tuy nhiên file đồ án không quy định xuống dòng chính thức, sinh viên tự tùy chỉnh 3 hoặc 6pt cho phù hợp
\setlength{\parindent}{15pt} %Lùi đầu dòng 15pt.

\usepackage{outlines}

%\usepackage{latexsym} % Các kí hiệu toán học
%\usepackage{amsbsy} % Hỗ trợ các kí hiệu in đậm
%\usepackage{times} % Chọn font Time New Romans
%\usepackage{chngcntr} % Dùng để thiết lập lại cách đánh số caption,..
%\usepackage{parskip}
%\usepackage{tocbasic}
%\usepackage{blindtext}

% \includeonly{chapters/1_Khao_sat}

\begin{document}

%!TeX root = main.tex
\newcommand{\underwrite}[3][]{% \underwrite[<thickness>]{<numerator>}{<denominator>}
	\genfrac{}{}{#1}{}{\textstyle #2}{\scriptstyle #3}
}

\begin{titlepage}
	\thispagestyle{empty}
	% \thisfancypage{
	% \setlength{\fboxsep}{3pt}
	% \fbox}{}
	\begin{center}

		{\textbf{\large{ĐẠI HỌC BÁCH KHOA HÀ NỘI}}}\\
		{\textbf{\large{KHOA TOÁN - TIN}}}\\[4cm]

		{\textbf{\huge{BÁO CÁO CUỐI KỲ}}}\\[1cm]
		{\textbf{\Large{\projecttitle}}}\\[1cm]

		\begin{table}[H]
			\centering
			\renewcommand{\arraystretch}{1.3}
			\begin{tabular}{|c|l|l|c|}
				\hline
				\textbf{STT} & \textbf{Họ và tên} & \textbf{Email}               & \textbf{MSSV} \\ \hline
				1            & NGUYỄN TRƯỜNG SƠN  & son.nt227148@sis.hust.edu.vn & 20227148      \\ \hline
				2            & HỌ TÊN SV 2        & email2@sis.hust.edu.vn       & 2022xxxx      \\ \hline
				3            & HỌ TÊN SV 3        & email3@sis.hust.edu.vn       & 2022xxxx      \\ \hline
				4            & HỌ TÊN SV 4        & email4@sis.hust.edu.vn       & 2022xxxx      \\ \hline
				5            & HỌ TÊN SV 5        & email5@sis.hust.edu.vn       & 2022xxxx      \\ \hline
				6            & HỌ TÊN SV 6        & email6@sis.hust.edu.vn       & 2022xxxx      \\ \hline
			\end{tabular}
		\end{table}
		\vspace{6cm}
		\textbf{HÀ NỘI, 10/2025}

	\end{center}

\end{titlepage}
 % Phần bìa
\pagestyle{plain} % Chỉ hiển thị số trang ở cuối trang

\newpage
\pagenumbering{gobble}

% ===================================================
% --- CÁC TRANG DANH MỤC ---
% Mục lục, Danh mục hình vẽ/bảng biểu vẫn dùng số La Mã
\newpage
\renewcommand*\contentsname{MỤC LỤC}

\titlecontents{chapter}
[0.0cm]             % left margin
{\bfseries\vspace{0.3cm}}                  % above code
{{\bfseries{\scshape}
			CHƯƠNG \thecontentslabel.\ }}
% numbered format
{}         % unnumbered format
{\titlerule*[0.3pc]{.}\contentspage}         % filler-page-format, e.g dots

\titlecontents{section}
[0.0cm]             % left margin
{\vspace{0.3cm}}                  % above code
{\thecontentslabel \ } % numbered format
{}         % unnumbered format
{\titlerule*[0.3pc]{.}\contentspage}         % filler-page-format, e.g dots

\titlecontents{subsection}
[1.0cm]             % left margin
{\vspace{0.3cm}}                  % above code
{\thecontentslabel \ } % numbered format
{}         % unnumbered format
{\titlerule*[0.3pc]{.}\contentspage}         % filler-page-format, e.g dots

\tableofcontents
\thispagestyle{empty}
\cleardoublepage

%Tạo danh mục hình vẽ.
\newpage
\renewcommand{\listfigurename}{DANH MỤC HÌNH VẼ}
{\let\oldnumberline\numberline
	\renewcommand{\numberline}{Hình~\oldnumberline}
	\listoffigures}
\phantomsection\addcontentsline{toc}{section}{\numberline {} DANH MỤC HÌNH VẼ}

%Tạo danh mục bảng biểu.
\renewcommand{\listtablename}{DANH MỤC BẢNG BIỂU}
{\let\oldnumberline\numberline
	\renewcommand{\numberline}{Bảng~\oldnumberline}
	\listoftables}
\phantomsection\addcontentsline{toc}{section}{\numberline {} DANH MỤC BẢNG BIỂU}

%Mục dưới đây dành cho danh mục thuật ngữ và viết tắt, chỉ sử dụng khi có từ viết tắt
\glsaddall
\renewcommand*{\glossaryname}{Danh sách thuật ngữ}
\renewcommand*{\acronymname}{DANH MỤC THUẬT NGỮ VÀ TỪ VIẾT TẮT}
\renewcommand*{\entryname}{Thuật ngữ}
\renewcommand*{\descriptionname}{Ý nghĩa}
\printnoidxglossaries
\phantomsection\addcontentsline{toc}{section}{\numberline {} DANH MỤC THUẬT NGỮ VÀ TỪ VIẾT TẮT}

%\mainmatter % Lệnh này tự động chuyển sang số Ả Rập (1, 2, 3...) và reset bộ đếm

\newpage
\pagenumbering{arabic}

\pagestyle{fancy}
\fancyhf{}
\fancyhead[RE, LO]{\leftmark}
%\fancyhead[LE]{\rightmark}
\fancyfoot[RE, LO]{\thepage}
\renewcommand{\headrulewidth}{0.4pt}

% !TEX root = ../main.tex
\chapter*{BẢNG PHÂN CÔNG NHIỆM VỤ}
\addcontentsline{toc}{chapter}{BẢNG PHÂN CÔNG NHIỆM VỤ}

\begin{table}[H]
  \centering
  \renewcommand{\arraystretch}{1.5}
  \setlength{\tabcolsep}{5pt}
  \begin{tabular}{|c|p{4.5cm}|c|p{5cm}|c|}
    \hline
    \textbf{STT} & \textbf{Họ và tên} & \textbf{MSSV} & \textbf{Nhiệm vụ được giao} & \textbf{Đánh giá} \\
    \hline

    % Thành viên 1
    1 & Nguyễn Trường Sơn & 20227148 &
    \begin{itemize}[nosep, leftmargin=1em, before=\vspace{-0.5\baselineskip}]
      \item Nhóm trưởng (Leader)
      \item Quản lý tiến độ dự án
      \item Thiết kế Biểu đồ ca (usecase), Biểu đồ luồng, Biểu đồ ERD
      \item Tổng hợp báo cáo
    \end{itemize} & A+ \\
    \hline

    % Thành viên 2
    2 & Trần Bảo Châu & 20216799 & Thiết kế sơ đồ phân tích lớp & B \\
    \hline

    % Thành viên 3
    3 & Đinh Minh Hà & 20227194 &
    \begin{itemize}[nosep, leftmargin=1em, before=\vspace{-0.5\baselineskip}]
      \item Thiết kế giao diện, Phân tích các biểu đồ ca quản lý
      \item Vẽ sơ đồ chuyển màn hình
    \end{itemize} & A \\
    \hline

    % Thành viên 4
    4 & Nguyễn Doanh Thái & 20237483 &
    \begin{itemize}[nosep, leftmargin=1em, before=\vspace{-0.5\baselineskip}]
      \item Thiết kế giao diện, Phân tích các biểu đồ ca quản lý
      \item Vẽ sơ đồ chuyển màn hình
    \end{itemize} & A \\
    \hline

    % Thành viên 5
    5 & Cao Phạm Minh Tuấn & 20237492 & Phân tích các biểu đồ ca nghiệp vụ bán hàng (POS) & A \\
    \hline

    % Thành viên 6
    6 & Phạm Xuân Vỹ & 20237496 &
    Phân tích các biểu đồ ca nghiệp vụ kho & A \\
    \hline
  \end{tabular}
  \caption*{Bảng tổng hợp phân công công việc}
\end{table}

\vspace{1cm}
\textbf{Cam kết của nhóm:} Các thành viên đã hoàn thành đầy đủ nhiệm vụ được giao đúng thời hạn và đảm bảo chất lượng công việc.

\newpage

\chapter{LỜI NÓI ĐẦU}

Mô hình cửa hàng tiện lợi (CVS - Convenience Store) như Circle K, GS25 và 7-11 rất phổ biến và được ưa chuộng với nhóm học sinh, sinh viên bởi đặc tính tiện lợi, mở 24/7 và cung cấp chỗ ngồi. Không chỉ bán đồ tạp hoá, CVS còn cung cấp cả thực phẩm chế biến như mì trộn, bánh bao hoặc đồ uống như Milo đá, Coca tươi. Bởi vậy nên bên cạnh các nghiệp vụ quản lý truyền thống của một cửa hàng như POS hay quản lý nhân sự, CVS còn bao gồm các nghiệp vụ đặc biệt của một cửa hàng đồ ăn nhanh, ví dụ như chế biến hay quản lý hạn sử dụng trong ngày thay vì quản lý theo ngày/tháng.

Trong báo cáo này, chúng em sẽ trình bày chi tiết các bước thực hiện dự án, từ phân tích yêu cầu, thiết kế hệ thống, đến triển khai và kiểm thử Hệ thống Quản lý CVS. Các kỹ thuật và công nghệ đã được áp dụng để đảm bảo hệ thống hoạt động chính xác, ổn định và có thể truy vết. Đặc biệt, Hệ thống quản lý CVS được xây dựng để có thể hoạt động liên tục trong thời gian dài, hỗ trợ nhân viên phục vụ hàng trăm đơn hàng trong giờ cao điểm mà không gặp phải sự cố hay gây khó khăn cho nhân viên sử dụng. 

Chúng em hy vọng rằng, với Hệ thống Quản lý CVS, các chủ cửa hàng CVS sẽ có thêm lựa chọn đơn giản mà mạnh mẽ để quản lý một mô hình rất mới, rất độc đáo và tiềm năng, cạnh tranh với các chuỗi CVS lớn. Với dự án này, chúng em mong rằng sẽ góp phần thúc đẩy doanh số ngành bán lẻ. 

Chúng em xin chân thành cảm ơn sự hỗ trợ và hướng dẫn từ thầy Lê Hải Hà, cùng sự hợp tác của các thành viên trong nhóm, để dự án có thể hoàn thành đúng tiến độ và đạt được những mục tiêu đề ra.


% \label{chapter:Introduction}
% !TEX root = ../main.tex
\chapter{KHẢO SÁT BÀI TOÁN}

\section{Mô tả yêu cầu bài toán}
% Phần này sẽ cần 2 mục: Yêu cầu nghiệp vụ, và phân rã chức năng.

\section{Thông tin nghiệp vụ cơ bản}
% 1 bảng 4 cột: Tên nghiệp vụ, quy trình, kết quả, điều kiện

 % Phần mở đầu

\newpage
%\pagestyle{fancy} % Áp dụng header và footer
% \label{chapter:Related_works}
% !TEX root = ../main.tex
\chapter{ĐẶC TẢ YÊU CẦU}

\section{Biểu đồ ca sử dụng}

\subsection{Các tác nhân của hệ thống}

Hệ thống Cửa hàng tiện lợi (CVS) bao gồm các tác nhân sau đây:

\begin{longtable}{|c|p{3cm}|p{10cm}|}
	\hline
	\textbf{Mã} & \textbf{Tên tác nhân} & \textbf{Mô tả}                                                                                                                                                          \\
	\hline
	\endfirsthead

	\hline
	\textbf{Mã} & \textbf{Tên tác nhân} & \textbf{Mô tả}                                                                                                                                                          \\
	\hline
	\endhead

	\hline
	\endfoot

	\hline
	\caption{Danh sách các tác nhân trong hệ thống} \label{tab:actors}                                                                                                                                            \\
	\endlastfoot

	AC01        & Nhân viên             & Thực hiện các nghiệp vụ bán hàng, nhập hàng, kiểm kê, quản lý ca làm việc, ...                                                                                          \\
	\hline
	AC02        & Quản lý               & Là nhân viên có quyền hạn cao hơn, có thể thực hiện các chức năng quản trị như quản lý nhân viên, cấu hình hệ thống, xem báo cáo thống kê và quản lý danh mục sản phẩm. \\
	\hline
	AC03        & Khách hàng            & Người mua hàng tại cửa hàng, tương tác với hệ thống thông qua quá trình thanh toán và nhận hóa đơn.                                                                     \\                                                                            \\
	\hline
	AC04        & Máy in nhiệt          & Thiết bị phần cứng dùng để in hóa đơn bán hàng và phiếu chế biến thực phẩm.                                                                                             \\
	\hline
	AC05        & Nhà cung cấp          & Đối tác bên ngoài cung cấp hàng hóa cho cửa hàng, tham gia vào quy trình nhập hàng.                                                                                     \\
	\hline
\end{longtable}

\subsection{Các ca sử dụng}

Dựa trên phân tích yêu cầu, hệ thống CVS có các ca sử dụng như sau:

\begin{longtable}{|c|p{3cm}|p{6cm}|p{3.5cm}|}
	\hline
	\textbf{Mã} & \textbf{Tên ca sử dụng} & \textbf{Mô tả ngắn}                                                                               & \textbf{Tác nhân}                   \\
	\hline
	\endfirsthead

	\hline
	\textbf{Mã} & \textbf{Tên ca sử dụng} & \textbf{Mô tả ngắn}                                                                               & \textbf{Tác nhân}                   \\
	\hline
	\endhead

	\hline
	\endfoot

	\hline
	\caption{Danh sách các ca sử dụng trong hệ thống} \label{tab:usecases}                                                                                                          \\
	\endlastfoot

	UC01        & Đăng nhập               & Xác thực người dùng để truy cập hệ thống bằng mã PIN.                                             & Nhân viên                           \\
	\hline
	UC02        & Đăng xuất               & Kết thúc phiên làm việc và thoát khỏi hệ thống.                                                   & Nhân viên                           \\
	\hline
	UC03        & Đặt lại PIN             & Thay đổi mã PIN đăng nhập của mình.                                                               & Nhân viên                           \\
	\hline
	UC04        & Bán hàng                & Thực hiện quy trình bán hàng: chọn sản phẩm, tính tiền, thanh toán và xuất hóa đơn.               & Nhân viên, Khách hàng, Máy in nhiệt \\
	\hline
	UC05        & Chế biến thực phẩm      & Chế biến tại chỗ (đồ ăn nóng, thức uống) phát sinh từ đơn hàng hoặc duy trì                       & Nhân viên, Khách hàng               \\
	\hline
	UC06        & Làm lại đơn hàng        & Sửa đơn hàng sau thanh toán khi có sự cố (hàng hỏng, sai món, ...).                               & Nhân viên                           \\
	\hline
	UC07        & Tiêu huỷ                & Huỷ bỏ sản phẩm bị lỗi, hết hạn hoặc hư hỏng.                                                     & Nhân viên                           \\
	\hline
	UC08        & Nhập hàng               & Nhập hàng hóa mới vào kho, cập nhật số lượng tồn kho.                                             & Nhân viên, Nhà cung cấp             \\
	\hline
	UC09        & Kiểm kê tồn kho         & Kiểm tra và đối chiếu số lượng hàng thực tế với dữ liệu trong hệ thống.                           & Nhân viên                           \\
	\hline
	UC10        & Quản lý ca làm việc     & Thực hiện quy trình Mở/đóng ca gồm ghi nhận doanh thu và kiểm đếm tiền quỹ.                       & Nhân viên                           \\
	\hline
	UC11        & Quản lý công thức       & Tạo, sửa, xóa công thức chế biến cho các sản phẩm.                                                & Quản lý                             \\
	\hline
	UC12        & Quản lý combo           & Thiết lập các gói combo sản phẩm.                                                                 & Quản lý                             \\
	\hline
	UC13        & Quản lý danh mục hàng   & Quản lý danh mục sản phẩm (trong kho, sản phẩm bán), thông tin hàng hóa, giá bán.                 & Quản lý                             \\
	\hline
	UC14        & Quản lý khuyến mại      & Tạo và quản lý các chương trình khuyến mại, giảm giá.                                             & Quản lý                             \\
	\hline
	UC15        & Quản lý nhân viên       & Thêm, sửa, xóa thông tin nhân viên và phân quyền truy cập.                                        & Quản lý                             \\
	\hline
	UC16        & Cấu hình hệ thống       & Thiết lập các thông số hệ thống: thông tin cửa hàng, máy in, thuế suất, tài khoản thanh toán, ... & Quản lý                             \\
	\hline
	UC17        & Báo cáo thống kê        & Xem báo cáo doanh thu, lợi nhuận, hàng bán chạy và các thống kê khác.                             & Quản lý                             \\
	\hline
	UC18        & Chấm công               & Ghi nhận thời gian vào/ra làm việc của nhân viên.                                                 & Nhân viên                           \\
	\hline
	UC19        & Tính công               & Tính toán và xem tổng hợp giờ công, ngày công làm việc trong kỳ.                                  & Nhân viên                           \\
	\hline
	UC20        & Xin nghỉ                & Gửi yêu cầu xin nghỉ phép, nghỉ ốm hoặc các loại nghỉ khác.                                       & Nhân viên, Quản lý                  \\
	\hline
	UC21        & Đăng ký lịch làm việc   & Đăng ký ca làm việc mong muốn trong tuần/tháng tới.                                               & Nhân viên                           \\
	\hline
\end{longtable}

\subsection{Quan hệ giữa các ca sử dụng}

Các ca sử dụng trong hệ thống có các mối quan hệ sau:

\begin{longtable}{|l|p{4.5cm}|p{7cm}|}
	\hline
	\textbf{Loại quan hệ} & \textbf{Ca sử dụng}                      & \textbf{Mô tả}                                                        \\
	\hline
	\endfirsthead

	\hline
	\textbf{Loại quan hệ} & \textbf{Ca sử dụng}                      & \textbf{Mô tả}                                                        \\
	\hline
	\endhead

	\hline
	\endfoot

	\hline
	\caption{Quan hệ giữa các ca sử dụng} \label{tab:uc-relations}                                                                           \\
	\endlastfoot

	Extend                & Chế biến thực phẩm $\leftarrow$ Bán hàng & Khi đơn hàng có sản phẩm chế biến thì kích hoạt quy trình chế biến.   \\
	\hline
	Extend                & Tiêu huỷ $\leftarrow$ Làm lại đơn hàng   & Khi làm lại đơn hàng có thể cần tiêu huỷ đơn cũ.                      \\
	\hline
	Generalization        & Quản lý $\rightarrow$ Nhân viên          & Quản lý kế thừa tất cả quyền của Nhân viên và có thêm quyền quản trị. \\
	\hline
\end{longtable}

\vfill

\subsection{Biểu đồ ca sử dụng tổng quan}

\begin{figure}[H]
	\centering
	\includegraphics[width=\textwidth]{assets/diagram_usecase.png}
	\caption{Biểu đồ ca sử dụng Hệ thống CVS}
	\label{fig:usecase-diagram}
\end{figure}

\clearpage

\section{Đặc tả chi tiết các ca sử dụng}

Phần này trình bày đặc tả chi tiết cho các ca sử dụng quan trọng nhất của hệ thống.

\subsection{Đặc tả ca sử dụng Bán hàng}

\textbf{Ca sử dụng:} UC04 - Bán hàng \\
\textbf{Tiền điều kiện:} \\
Nhân viên đã đăng nhập vào hệ thống.

\vspace{0.5em}
\noindent \textbf{Luồng sự kiện chính:}
\begin{enumerate}
	\item \textbf{Nhân viên}: Chọn chức năng bán hàng trên giao diện chính.
	\item \textbf{Hệ thống}: Hiển thị giao diện bán hàng, sẵn sàng quét mã hoặc tìm kiếm sản phẩm.
	\item \textbf{Nhân viên}: Quét mã vạch sản phẩm hoặc tìm kiếm theo tên.
	\item \textbf{Hệ thống}: Thêm sản phẩm vào giỏ hàng, hiển thị tên, số lượng, đơn giá và thành tiền. Nếu có khuyến mại, hệ thống tự động áp dụng.
	\item \textbf{Nhân viên}: Điều chỉnh số lượng sản phẩm nếu khách mua nhiều (tùy chọn).
	\item \textbf{Hệ thống}: Cập nhật lại tổng tiền đơn hàng.
	\item \textbf{Nhân viên}: Nhấn nút "Thanh toán" khi đã nhập đủ sản phẩm.
	\item \textbf{Hệ thống}: Hiển thị màn hình chọn phương thức thanh toán (Tiền mặt, Chuyển khoản, Thẻ, Ví điện tử).
	\item \textbf{Khách hàng}: Thanh toán tiền cho đơn hàng.
	\item \textbf{Nhân viên}: Xác nhận thanh toán thành công trên hệ thống.
	\item \textbf{Hệ thống}: Lưu thông tin đơn hàng, cập nhật trừ kho và gửi lệnh in hóa đơn.
	\item \textbf{Máy in}: In hóa đơn cho khách hàng.
\end{enumerate}

\vspace{0.5em}
\noindent \textbf{Luồng sự kiện thay thế:}
\begin{description}
	\item[4a.] \textbf{Không tìm thấy sản phẩm:} Hệ thống thông báo không tìm thấy mã vạch. Nhân viên tìm kiếm thủ công bằng tên hoặc chọn từ danh mục. Quay lại bước 4.
	\item[9a.] \textbf{Thanh toán thất bại:} Cổng thanh toán báo lỗi (thẻ lỗi, ví không đủ tiền). Hệ thống thông báo lỗi. Nhân viên yêu cầu khách chọn phương thức thanh toán khác. Quay lại bước 8.
	\item[7a.] \textbf{Hủy đơn hàng:} Khách không mua nữa. Nhân viên chọn "Hủy đơn". Hệ thống xóa giỏ hàng hiện tại. Ca sử dụng kết thúc.
\end{description}

\vspace{0.5em}
\noindent \textbf{Hậu điều kiện:} \\
Đơn hàng được lưu vào lịch sử, kho hàng được cập nhật giảm số lượng, doanh thu được ghi nhận cho ca làm việc hiện tại.

\subsection{Đặc tả ca sử dụng Nhập hàng}

\textbf{Ca sử dụng:} UC08 - Nhập hàng \\
\textbf{Tiền điều kiện:} \\
Nhân viên đã đăng nhập hệ thống. Thông tin nhà cung cấp đã được cấu hình trong hệ thống.

\vspace{0.5em}
\noindent \textbf{Luồng sự kiện chính:}
\begin{enumerate}
	\item \textbf{Nhân viên}: Chọn chức năng nhập hàng từ menu.
	\item \textbf{Hệ thống}: Hiển thị danh sách phiếu nhập và nút tạo phiếu nhập mới.
	\item \textbf{Nhân viên}: Chọn "Tạo phiếu nhập mới", chọn nhà cung cấp.
	\item \textbf{Nhân viên}: Quét mã hoặc tìm kiếm sản phẩm cần nhập.
	\item \textbf{Hệ thống}: Hiển thị thông tin sản phẩm và ô nhập số lượng, giá nhập.
	\item \textbf{Nhân viên}: Nhập số lượng dự kiến và giá nhập.
	\item \textbf{Hệ thống}: Thêm sản phẩm vào danh sách nhập, tự động tính tổng tiền phiếu nhập.
	\item \textbf{Nhân viên}: Lặp lại bước 4-7 cho đến khi hoàn tất danh sách.
	\item \textbf{Nhân viên}: Nhấn "Lưu phiếu nhập" để tạo phiếu nhập với trạng thái "Chờ giao".
	\item \textbf{Hệ thống}: Lưu phiếu nhập, gửi yêu cầu đến Nhà cung cấp.
	\item \textbf{Nhà cung cấp}: Giao hàng theo phiếu nhập.
	\item \textbf{Nhân viên}: Kiểm tra hàng thực tế nhận được, đối chiếu với phiếu nhập.
	\item \textbf{Nhân viên}: Xác nhận số lượng thực nhận trên giao diện.
	\item \textbf{Nhân viên}: Nhấn "Xác nhận nhập kho".
	\item \textbf{Hệ thống}: Cập nhật tồn kho theo số lượng thực nhận, lưu lịch sử nhập hàng kèm ghi chú thay đổi (nếu có).
\end{enumerate}

\vspace{0.5em}
\noindent \textbf{Luồng sự kiện thay thế:}
\begin{description}
	\item[4a.] \textbf{Sản phẩm chưa có trong hệ thống:} Hệ thống thông báo không tìm thấy sản phẩm. Nhân viên chọn "Tạo sản phẩm mới", nhập thông tin (tên, giá bán, danh mục). Hệ thống lưu sản phẩm mới. Quay lại bước 5.
	\item[12a.] \textbf{Số lượng chênh lệch với phiếu nhập ban đầu:} Hệ thống hiển thị giao diện nhập số lượng thực tế và ghi chú lý do (hàng hỏng, thiếu, ...). Nhân viên nhập thông tin. Quay lại bước 13.
	\item[12b.] \textbf{Hàng không đúng hoàn toàn:} Nhà cung cấp giao sai hàng hoặc hàng hỏng toàn bộ. Nhân viên từ chối nhận, ghi chú lý do. Hệ thống cập nhật trạng thái phiếu nhập thành "Từ chối". Ca sử dụng kết thúc.
\end{description}

\vspace{0.5em}
\noindent \textbf{Hậu điều kiện:} \\
Số lượng tồn kho được cập nhật theo số lượng thực nhận. Hệ thống ghi nhận phiếu nhập, chi phí nhập hàng và lịch sử thay đổi (nếu có chênh lệch).

\subsection{Đặc tả ca sử dụng Quản lý ca làm việc}

\textbf{Ca sử dụng:} UC10 - Quản lý ca làm việc \\
\textbf{Tiền điều kiện:} \\
Nhân viên đã đăng nhập hệ thống. Ca làm việc trước đó đã được đóng (hoặc đây là ca đầu tiên).

\vspace{0.5em}
\noindent \textbf{Luồng sự kiện chính:}
\begin{enumerate}
	\item \textbf{Hệ thống}: Yêu cầu xác nhận mở ca (Open Shift) và khai báo tiền đầu ca (Tiền quỹ).
	\item \textbf{Nhân viên}: Đếm tiền thực tế trong két và nhập số tiền đầu ca.
	\item \textbf{Hệ thống}: Ghi nhận trạng thái mở ca và bắt đầu phiên làm việc.
	\item \textbf{Nhân viên}: Thực hiện các hoạt động bán hàng trong ca.
	\item \textbf{Nhân viên}: Chọn chức năng "Kết ca" (Close Shift) khi hết giờ làm.
	\item \textbf{Hệ thống}: Hiển thị bảng tổng kết doanh thu dự kiến (tiền mặt, thẻ, chuyển khoản) dựa trên các đơn hàng đã bán.
	\item \textbf{Nhân viên}: Đếm tiền thực tế trong két và nhập vào hệ thống.
	\item \textbf{Hệ thống}: So sánh tiền thực tế và tiền dự kiến (System expected). Hiển thị chênh lệch (thừa/thiếu) nếu có.
	\item \textbf{Nhân viên}: Nhập lý do chênh lệch (nếu có) và xác nhận kết ca.
	\item \textbf{Hệ thống}: Lưu báo cáo ca làm việc.
\end{enumerate}

\vspace{0.5em}
\noindent \textbf{Hậu điều kiện:} \\
Ca làm việc được đóng lại. Báo cáo doanh thu, tiền mặt, chênh lệch được lưu trữ để Quản lý đối soát.

\subsection{Đặc tả ca sử dụng Đăng nhập}

\textbf{Ca sử dụng:} UC01 - Đăng nhập \\
\textbf{Tiền điều kiện:} \\
Nhân viên đã có tài khoản

\vspace{0.5em}
\noindent \textbf{Luồng sự kiện chính:}
\begin{enumerate}
	\item \textbf{Nhân viên}: Mở ứng dụng bán hàng trên thiết bị POS.
	\item \textbf{Hệ thống}: Hiển thị giao diện yêu cầu nhập mã PIN.
	\item \textbf{Nhân viên}: Nhập mã PIN cá nhân.
	\item \textbf{Hệ thống}: Xác thực thông tin đăng nhập.
	\item \textbf{Hệ thống}: Nếu thông tin đúng, chuyển hướng vào màn hình chính với quyền hạn tương ứng.
\end{enumerate}

\vspace{0.5em}
\noindent \textbf{Luồng sự kiện thay thế:}
\begin{description}
	\item[4a.] \textbf{Mã PIN không đúng:} Hệ thống hiển thị thông báo "Đăng nhập thất bại". Nhân viên nhập lại mã PIN. Quay lại bước 4.
	\item[4b.] \textbf{Quên mã PIN:} Hệ thống hiển thị hộp thoại yêu cầu nhập e-mail đăng ký. Nhân viên nhập địa chỉ e-mail, hệ thống gửi mã PIN mới vào hòm thư. Qauy lại bước 3.
\end{description}

\vspace{0.5em}
\noindent \textbf{Hậu điều kiện:} \\
Nhân viên truy cập được vào hệ thống với quyền hạn đã được cấp.

\subsection{Đặc tả ca sử dụng Chế biến thực phẩm}

\textbf{Ca sử dụng:} UC05 - Chế biến thực phẩm \\
\textbf{Tiền điều kiện:} \\
Nhân viên đã đăng nhập hệ thống. Công thức chế biến đã được cấu hình. Nguyên liệu có sẵn trong kho.

\vspace{0.5em}
\noindent \textbf{Luồng sự kiện chính (Chế biến từ đơn hàng):}
\begin{enumerate}
	\item \textbf{Hệ thống}: Nhận đơn hàng có sản phẩm cần chế biến từ UC04 - Bán hàng.
	\item \textbf{Máy in nhiệt}: In phiếu chế biến (phiếu bếp/bar) với thông tin món và số lượng.
	\item \textbf{Nhân viên}: Nhận phiếu, thực hiện chế biến theo công thức.
	\item \textbf{Nhân viên}: Hoàn thành chế biến, thu hồi phiếu chế biến.
	\item \textbf{Nhân viên}: Giao sản phẩm cho khách hàng.
\end{enumerate}

\vspace{0.5em}
\noindent \textbf{Luồng sự kiện thay thế:}
\begin{description}
	\item[1a.] \textbf{Chế biến để fill đồ (duy trì tồn kho):} Nhân viên chọn chức năng "Chế biến" từ menu. Hệ thống hiển thị danh sách sản phẩm có thể chế biến. Nhân viên chọn sản phẩm và nhập số lượng cần chế biến. Hệ thống kiểm tra nguyên liệu, tự động trừ kho nguyên liệu theo công thức. Nhân viên xác nhận hoàn thành. Hệ thống cập nhật tăng tồn kho sản phẩm chế biến.
	\item[3a.] \textbf{Thiếu nguyên liệu:} Nhân viên phát hiện thiếu nguyên liệu khi chế biến. Nhân viên thông báo cho khách và đề xuất món thay thế hoặc hoàn tiền. Quay lại UC04 để xử lý.
\end{description}

\vspace{0.5em}
\noindent \textbf{Hậu điều kiện:} \\
Sản phẩm được chế biến và giao cho khách (hoặc cập nhật tồn kho nếu fill đồ). Nguyên liệu được trừ theo công thức.

\subsection{Đặc tả ca sử dụng Làm lại đơn hàng}

\textbf{Ca sử dụng:} UC06 - Làm lại đơn hàng \\
\textbf{Tiền điều kiện:} \\
Nhân viên đã đăng nhập hệ thống. Đơn hàng cần làm lại đã tồn tại trong lịch sử.

\vspace{0.5em}
\noindent \textbf{Luồng sự kiện chính:}
\begin{enumerate}
	\item \textbf{Khách hàng}: Phàn nàn về sản phẩm (hỏng, sai món, không đúng yêu cầu, ...).
	\item \textbf{Nhân viên}: Tiếp nhận phàn nàn, chọn chức năng "Làm lại đơn hàng".
	\item \textbf{Nhân viên}: Tìm kiếm và chọn đơn hàng cần làm lại từ lịch sử.
	\item \textbf{Hệ thống}: Hiển thị chi tiết đơn hàng và các sản phẩm.
	\item \textbf{Nhân viên}: Chọn sản phẩm cần làm lại, nhấn "Làm lại".
	\item \textbf{Hệ thống}: Hiển thị form yêu cầu nhập lý do làm lại và thông tin nguyên liệu thất thoát.
	\item \textbf{Nhân viên}: Nhập lý do và xác nhận nguyên liệu bị hao hụt.
	\item \textbf{Nhân viên}: Nhấn "Xác nhận làm lại".
	\item \textbf{Hệ thống}: Tự động tạo phiếu tiêu hủy cho sản phẩm lỗi (UC07), trừ nguyên liệu thất thoát.
	\item \textbf{Máy in nhiệt}: In phiếu chế biến mới.
	\item \textbf{Nhân viên}: Thực hiện chế biến lại sản phẩm.
	\item \textbf{Nhân viên}: Giao sản phẩm mới cho khách hàng.
	\item \textbf{Nhân viên}: Nhấn "Hoàn tất" trên hệ thống.
	\item \textbf{Hệ thống}: Lưu lịch sử làm lại đơn hàng, ghi nhận nguyên liệu thất thoát.
\end{enumerate}

\vspace{0.5em}
\noindent \textbf{Luồng sự kiện thay thế:}
\begin{description}
	\item[5a.] \textbf{Hoàn tiền thay vì làm lại:} Khách yêu cầu hoàn tiền. Nhân viên chọn "Hoàn tiền", nhập lý do. Hệ thống xử lý hoàn tiền và ghi nhận. Ca sử dụng kết thúc.
\end{description}

\vspace{0.5em}
\noindent \textbf{Hậu điều kiện:} \\
Sản phẩm mới được giao cho khách. Lịch sử làm lại và nguyên liệu thất thoát được ghi nhận để báo cáo.

\subsection{Đặc tả ca sử dụng Tiêu huỷ}

\textbf{Ca sử dụng:} UC07 - Tiêu huỷ \\
\textbf{Tiền điều kiện:} \\
Nhân viên đã đăng nhập hệ thống.

\vspace{0.5em}
\noindent \textbf{Luồng sự kiện chính:}
\begin{enumerate}
	\item \textbf{Nhân viên}: Chọn chức năng "Tiêu huỷ" từ menu.
	\item \textbf{Hệ thống}: Hiển thị giao diện tiêu huỷ với danh sách sản phẩm.
	\item \textbf{Nhân viên}: Tìm kiếm và chọn sản phẩm cần tiêu huỷ.
	\item \textbf{Nhân viên}: Nhập số lượng tiêu huỷ và lý do (hết hạn, hỏng, rơi vỡ, ...).
	\item \textbf{Nhân viên}: Nhấn "Xác nhận tiêu huỷ".
	\item \textbf{Hệ thống}: Hiển thị xác nhận tổng giá trị hàng tiêu huỷ.
	\item \textbf{Nhân viên}: Xác nhận lần cuối.
	\item \textbf{Hệ thống}: Trừ tồn kho, lưu lịch sử tiêu huỷ kèm lý do và người thực hiện.
\end{enumerate}

\vspace{0.5em}
\noindent \textbf{Hậu điều kiện:} \\
Tồn kho được cập nhật giảm. Lịch sử tiêu huỷ được ghi nhận để đối soát và báo cáo thất thoát.

\subsection{Đặc tả ca sử dụng Kiểm kê tồn kho}

\textbf{Ca sử dụng:} UC09 - Kiểm kê tồn kho \\
\textbf{Tiền điều kiện:} \\
Nhân viên đã đăng nhập hệ thống.

\vspace{0.5em}
\noindent \textbf{Luồng sự kiện chính:}
\begin{enumerate}
	\item \textbf{Nhân viên}: Chọn chức năng "Kiểm kê tồn kho" từ menu.
	\item \textbf{Hệ thống}: Hiển thị danh sách sản phẩm với số lượng tồn kho hiện tại theo hệ thống.
	\item \textbf{Nhân viên}: Chọn sản phẩm cần kiểm kê.
	\item \textbf{Hệ thống}: Hiển thị số lượng hệ thống và form nhập số lượng thực tế.
	\item \textbf{Nhân viên}: Đếm hàng thực tế, nhập số lượng vào form.
	\item \textbf{Hệ thống}: So sánh và hiển thị chênh lệch (nếu có).
	\item \textbf{Nhân viên}: Nhập lý do chênh lệch (nếu có): hao hụt, mất mát, nhập sai trước đó, ...
	\item \textbf{Nhân viên}: Nhấn "Xác nhận kiểm kê".
	\item \textbf{Hệ thống}: Cập nhật tồn kho theo số lượng thực tế, lưu lịch sử điều chỉnh kèm lý do.
\end{enumerate}

\vspace{0.5em}
\noindent \textbf{Hậu điều kiện:} \\
Tồn kho được cập nhật chính xác theo thực tế. Lịch sử kiểm kê và điều chỉnh được lưu để đối soát.

\subsection{Đặc tả ca sử dụng Chấm công}

\textbf{Ca sử dụng:} UC18 - Chấm công \\
\textbf{Tiền điều kiện:} \\
Nhân viên có tài khoản hợp lệ trong hệ thống.

\vspace{0.5em}
\noindent \textbf{Luồng sự kiện chính:}
\begin{enumerate}
	\item \textbf{Nhân viên}: Mở ứng dụng và đăng nhập hệ thống.
	\item \textbf{Hệ thống}: Hiển thị màn hình chính với nút "Clock-in" hoặc "Clock-out" tùy trạng thái.
	\item \textbf{Nhân viên}: Nhấn "Clock-in" khi bắt đầu làm việc.
	\item \textbf{Hệ thống}: Ghi nhận thời gian vào làm, hiển thị xác nhận.
	\item \textbf{Nhân viên}: Thực hiện công việc trong ca.
	\item \textbf{Nhân viên}: Nhấn "Clock-out" khi kết thúc làm việc.
	\item \textbf{Hệ thống}: Ghi nhận thời gian ra, tính tổng giờ làm trong ngày, hiển thị xác nhận.
\end{enumerate}

\vspace{0.5em}
\noindent \textbf{Luồng sự kiện thay thế:}
\begin{description}
	\item[3a.] \textbf{Quên clock-in:} Nhân viên quên chấm công đầu ca. Nhân viên báo Quản lý để điều chỉnh thủ công trong UC19 - Tính công.
\end{description}

\vspace{0.5em}
\noindent \textbf{Hậu điều kiện:} \\
Thời gian làm việc của nhân viên được ghi nhận để tính công.

\subsection{Đặc tả ca sử dụng Xin nghỉ}

\textbf{Ca sử dụng:} UC20 - Xin nghỉ \\
\textbf{Tiền điều kiện:} \\
Nhân viên đã đăng nhập hệ thống.

\vspace{0.5em}
\noindent \textbf{Luồng sự kiện chính:}
\begin{enumerate}
	\item \textbf{Nhân viên}: Chọn chức năng "Xin nghỉ" từ menu.
	\item \textbf{Hệ thống}: Hiển thị form xin nghỉ với các trường: ngày nghỉ, loại nghỉ (phép, ốm, việc riêng), lý do.
	\item \textbf{Nhân viên}: Điền thông tin và gửi yêu cầu.
	\item \textbf{Hệ thống}: Lưu yêu cầu với trạng thái "Chờ duyệt", thông báo đến Quản lý.
	\item \textbf{Quản lý}: Xem danh sách yêu cầu xin nghỉ, chọn yêu cầu cần xử lý.
	\item \textbf{Quản lý}: Phê duyệt hoặc từ chối yêu cầu, có thể ghi chú.
	\item \textbf{Hệ thống}: Cập nhật trạng thái yêu cầu, thông báo kết quả cho Nhân viên.
\end{enumerate}

\vspace{0.5em}
\noindent \textbf{Luồng sự kiện thay thế:}
\begin{description}
	\item[6a.] \textbf{Từ chối yêu cầu:} Quản lý từ chối với lý do (thiếu người, không hợp lệ, ...). Hệ thống thông báo cho Nhân viên. Ca sử dụng kết thúc.
\end{description}

\vspace{0.5em}
\noindent \textbf{Hậu điều kiện:} \\
Yêu cầu nghỉ được phê duyệt hoặc từ chối. Lịch làm việc được cập nhật tương ứng.

\subsection{Đặc tả ca sử dụng Đăng ký lịch làm việc}

\textbf{Ca sử dụng:} UC21 - Đăng ký lịch làm việc \\
\textbf{Tiền điều kiện:} \\
Nhân viên đã đăng nhập hệ thống. Quản lý đã tạo các ca làm việc khả dụng.

\vspace{0.5em}
\noindent \textbf{Luồng sự kiện chính:}
\begin{enumerate}
	\item \textbf{Nhân viên}: Chọn chức năng "Đăng ký lịch làm việc" từ menu.
	\item \textbf{Hệ thống}: Hiển thị lịch tuần/tháng với các ca làm việc khả dụng.
	\item \textbf{Nhân viên}: Chọn các ca muốn đăng ký.
	\item \textbf{Hệ thống}: Kiểm tra xung đột (đã đăng ký ca khác, vượt giờ làm tối đa, ...).
	\item \textbf{Hệ thống}: Hiển thị xác nhận đăng ký thành công.
	\item \textbf{Hệ thống}: Cập nhật lịch làm việc của nhân viên.
\end{enumerate}

\vspace{0.5em}
\noindent \textbf{Luồng sự kiện thay thế:}
\begin{description}
	\item[4a.] \textbf{Xung đột lịch:} Hệ thống phát hiện xung đột (đã có ca, vượt giờ). Hiển thị cảnh báo, yêu cầu chọn ca khác. Quay lại bước 3.
	\item[4b.] \textbf{Ca đã đầy:} Ca làm việc đã đủ số người đăng ký. Hệ thống thông báo và đề xuất ca khác. Quay lại bước 3.
\end{description}

\vspace{0.5em}
\noindent \textbf{Hậu điều kiện:} \\
Nhân viên được ghi nhận vào ca làm việc đã chọn. Lịch làm việc được cập nhật.


\newpage
%\pagestyle{fancy} % Áp dụng header và footer
% \label{chapter:Methodology}
% !TEX root = ../main.tex
\chapter{THIẾT KẾ KIẾN TRÚC}
Trong chương này, chúng ta tiến hành hiện thực hoá ca sử dụng (use cases) thành biểu đồ lớp phân tích và biểu đồ tuần tự (sequence diagram) để làm rõ các tương tác trong hệ thống.

\section{Thiết kế kiến trúc}
\subsection{Ca sử dụng UC01 - Đăng nhập}
\begin{figure}
  \centering
  \includegraphics[width=0.9\textwidth]{assets/diagram_analysis_sequence/SD_UC01_DangNhap.png}
  \caption{UC01 - Biểu đồ tuần tự Đăng nhập}
\end{figure}

\begin{figure}
  \centering
  \includegraphics[width=0.9\textwidth]{assets/diagram_analysis_class/ACD_UC01_DangNhap.png}
  \caption{UC01 - Biểu đồ lớp phân tích Đăng nhập}
\end{figure}

\subsection{Ca sử dụng UC04 - Bán hàng}
\begin{figure}
  \centering
  \includegraphics[width=0.9\textwidth]{assets/diagram_analysis_sequence/SD_UC04_BanHang.png}
  \caption{UC04 - Biểu đồ tuần tự Bán hàng}
\end{figure}

\begin{figure}
  \centering
  \includegraphics[width=0.9\textwidth]{assets/diagram_analysis_class/ACD_UC04_BanHang.png}
  \caption{UC04 - Biểu đồ lớp phân tích Bán hàng}
\end{figure}

\subsection{Ca sử dụng UC05 - Chế biến}
\begin{figure}
  \centering
  \includegraphics[width=0.9\textwidth]{assets/diagram_analysis_sequence/SD_UC05_CheBien.png}
  \caption{UC05 - Biểu đồ tuần tự Chế biến}
\end{figure}

\begin{figure}
  \centering
  \includegraphics[width=0.9\textwidth]{assets/diagram_analysis_class/ACD_UC05_CheBien.png}
  \caption{UC05 - Biểu đồ lớp phân tích Chế biến}
\end{figure}

\subsection{Ca sử dụng UC06 - Làm lại đơn}
\begin{figure}
  \centering
  \includegraphics[width=0.9\textwidth]{assets/diagram_analysis_sequence/SD_UC06_LamLaiDon.png}
  \caption{UC06 - Biểu đồ tuần tự Làm lại đơn}
\end{figure}

\begin{figure}
  \centering
  \includegraphics[width=0.9\textwidth]{assets/diagram_analysis_class/ACD_UC06_LamLaiDon.png}
  \caption{UC06 - Biểu đồ lớp phân tích Làm lại đơn}
\end{figure}

\subsection{Ca sử dụng UC07 - Tiêu hủy}
\begin{figure}
  \centering
  \includegraphics[width=0.9\textwidth]{assets/diagram_analysis_sequence/SD_UC07_TieuHuy.png}
  \caption{UC07 - Biểu đồ tuần tự Tiêu hủy}
\end{figure}

\begin{figure}
  \centering
  \includegraphics[width=0.9\textwidth]{assets/diagram_analysis_class/ACD_UC07_TieuHuy.png}
  \caption{UC07 - Biểu đồ lớp phân tích Tiêu hủy}
\end{figure}

\subsection{Ca sử dụng UC08 - Nhập hàng}
\begin{figure}
  \centering
  \includegraphics[width=0.9\textwidth]{assets/diagram_analysis_sequence/SD_UC08_NhapHang.png}
  \caption{UC08 - Biểu đồ tuần tự Nhập hàng}
\end{figure}

\begin{figure}
  \centering
  \includegraphics[width=0.9\textwidth]{assets/diagram_analysis_class/ACD_UC08_NhapHang.png}
  \caption{UC08 - Biểu đồ lớp phân tích Nhập hàng}
\end{figure}

\subsection{Ca sử dụng UC09 - Kiểm kê}
\begin{figure}
  \centering
  \includegraphics[width=0.9\textwidth]{assets/diagram_analysis_sequence/SD_UC09_KiemKe.png}
  \caption{UC09 - Biểu đồ tuần tự Kiểm kê}
\end{figure}

\begin{figure}
  \centering
  \includegraphics[width=0.9\textwidth]{assets/diagram_analysis_class/ACD_UC09_KiemKe.png}
  \caption{UC09 - Biểu đồ lớp phân tích Kiểm kê}
\end{figure}

\subsection{Ca sử dụng UC10 - Quản lý ca}
\begin{figure}
  \centering
  \includegraphics[width=0.9\textwidth]{assets/diagram_analysis_sequence/SD_UC10_QuanLyCa.png}
  \caption{UC10 - Biểu đồ tuần tự Quản lý ca}
\end{figure}

\begin{figure}
  \centering
  \includegraphics[width=0.9\textwidth]{assets/diagram_analysis_class/ACD_UC10_QuanLyCa.png}
  \caption{UC10 - Biểu đồ lớp phân tích Quản lý ca}
\end{figure}

\subsection{Ca sử dụng UC18 - Chấm công}
\begin{figure}
  \centering
  \includegraphics[width=0.9\textwidth]{assets/diagram_analysis_sequence/SD_UC18_ChamCong.png}
  \caption{UC18 - Biểu đồ tuần tự Chấm công}
\end{figure}

\begin{figure}
  \centering
  \includegraphics[width=0.9\textwidth]{assets/diagram_analysis_class/ACD_UC18_ChamCong.png}
  \caption{UC18 - Biểu đồ lớp phân tích Chấm công}
\end{figure}

\subsection{Ca sử dụng UC20 - Xin nghỉ}
\begin{figure}
  \centering
  \includegraphics[width=0.9\textwidth]{assets/diagram_analysis_sequence/SD_UC20_XinNghi.png}
  \caption{UC20 - Biểu đồ tuần tự Xin nghỉ}
\end{figure}

\begin{figure}
  \centering
  \includegraphics[width=0.9\textwidth]{assets/diagram_analysis_class/ACD_UC20_XinNghi.png}
  \caption{UC20 - Biểu đồ lớp phân tích Xin nghỉ}
\end{figure}

\subsection{Ca sử dụng UC21 - Đăng ký lịch}
\begin{figure}
  \centering
  \includegraphics[width=0.9\textwidth]{assets/diagram_analysis_sequence/SD_UC21_DangKyLich.png}
  \caption{UC21 - Biểu đồ tuần tự Đăng ký lịch}
\end{figure}

\begin{figure}
  \centering
  \includegraphics[width=0.9\textwidth]{assets/diagram_analysis_class/ACD_UC21_DangKyLich.png}
  \caption{UC21 - Biểu đồ lớp phân tích Đăng ký lịch}
\end{figure}

\section{Mô hình khái niệm}

\begin{figure}
  \centering
  \includegraphics[width=0.9\textwidth]{assets/diagram_class/Conceptual_Class_Diagram.png}
  \caption{Mô hình khái niệm hệ thống}
\end{figure}

\begin{figure}
  \centering
  \includegraphics[width=0.9\textwidth]{assets/diagram_class/Domain_Inventory.png}
  \caption{Mô hình khái niệm - Quản lý kho}
\end{figure}

\begin{figure}
  \centering
  \includegraphics[width=0.9\textwidth]{assets/diagram_class/Domain_Sales.png}
  \caption{Mô hình khái niệm - Bán hàng và thanh toán}
\end{figure}

\begin{figure}
  \centering
  \includegraphics[width=0.9\textwidth]{assets/diagram_class/Domain_UserAuth_Catalog.png}
  \caption{Mô hình khái niệm - Xác thực, quản lý ca và quản lý danh mục}
\end{figure}

\begin{figure}
  \centering
  \includegraphics[width=0.9\textwidth]{assets/diagram_class/Domain_WasteAudit_Config.png}
  \caption{Mô hình khái niệm - Theo dõi tiêu hủy và cấu hình hệ thống}
\end{figure}

\begin{figure}
  \centering
  \includegraphics[width=0.9\textwidth]{assets/diagram_class/Domain_Workforce.png}
  \caption{Mô hình khái niệm - Quản lý nhân sự}
\end{figure}

\section{Thiết kế hệ thống con}

\begin{figure}
  \centering
  \includegraphics[width=0.9\textwidth]{assets/diagram_subsystem/TongQuan.png}
  \caption{Tổng quan các hệ thống con}
\end{figure}

\begin{figure}
  \centering
  \includegraphics[width=0.9\textwidth]{assets/diagram_subsystem/AuthSubsystem.png}
  \caption{Hệ thống con Xác thực và Phân quyền}
\end{figure}

\begin{figure}
  \centering
  \includegraphics[width=0.9\textwidth]{assets/diagram_subsystem/POSSubsystem.png}
  \caption{Hệ thống con Bán hàng (POS)}
\end{figure}

\begin{figure}
  \centering
  \includegraphics[width=0.9\textwidth]{assets/diagram_subsystem/InventorySubsystem.png}
  \caption{Hệ thống con Quản lý kho}
\end{figure}

\begin{figure}
  \centering
  \includegraphics[width=0.9\textwidth]{assets/diagram_subsystem/CatalogSubsystem.png}
  \caption{Hệ thống con Quản lý danh mục}
\end{figure}

\begin{figure}
  \centering
  \includegraphics[width=0.9\textwidth]{assets/diagram_subsystem/ShiftSubsystem.png}
  \caption{Hệ thống con Quản lý ca làm việc}
\end{figure}

\begin{figure}
  \centering
  \includegraphics[width=0.9\textwidth]{assets/diagram_subsystem/AlertSubsystem.png}
  \caption{Hệ thống con Cảnh báo}
\end{figure}

\begin{figure}
  \centering
  \includegraphics[width=0.9\textwidth]{assets/diagram_subsystem/ReportSubsystem.png}
  \caption{Hệ thống con Báo cáo}
\end{figure}

\begin{figure}
  \centering
  \includegraphics[width=0.9\textwidth]{assets/diagram_subsystem/StoreSubsystem.png}
  \caption{Hệ thống con Quản lý cửa hàng}
\end{figure}


\newpage
%\pagestyle{fancy} % Áp dụng header và footer
% \label{chapter:Experiment}
% !TEX root = ../main.tex
\chapter{THIẾT KẾ CHƯƠNG TRÌNH}

\section{Thiết kế cơ sở dữ liệu}

\subsection{Tổng quan mô hình dữ liệu}

Hệ thống Quản lý Cửa hàng Tiện lợi (CVS) được thiết kế với 32 bảng dữ liệu, phân chia thành 8 miền nghiệp vụ chính. Mô hình dữ liệu được xây dựng theo các nguyên tắc:

\begin{itemize}
  \item \textbf{Chuẩn hóa dữ liệu:} Các bảng được thiết kế ở dạng chuẩn 3NF để tránh dư thừa và đảm bảo tính nhất quán.
  \item \textbf{Khả năng mở rộng đa cửa hàng:} Tất cả các bảng có phạm vi theo cửa hàng đều chứa khóa ngoại \texttt{store\_id}, sẵn sàng cho việc mở rộng hệ thống.
  \item \textbf{Truy vết đầy đủ:} Mọi thay đổi tồn kho đều được ghi nhận trong bảng \texttt{stock\_movements} với đầy đủ thông tin: ai thực hiện, thời điểm, lý do và lô bị ảnh hưởng.
  \item \textbf{Quản lý theo lô:} Hàng hóa được quản lý theo từng lô nhập với hạn sử dụng riêng biệt, hỗ trợ cơ chế FIFO (First-In-First-Out).
\end{itemize}

\subsection{Các miền dữ liệu}

Hệ thống được chia thành 8 miền dữ liệu chính, mỗi miền phục vụ một nhóm ca sử dụng cụ thể:

\begin{longtable}{|c|p{4cm}|p{6cm}}
  \hline
  \textbf{STT} & \textbf{Miền dữ liệu}  & \textbf{Các bảng}                                                                                                                                                                                              \\
  \hline
  \endfirsthead

  \hline
  \textbf{STT} & \textbf{Miền dữ liệu}  & \textbf{Các bảng}                                                                                                                                                                                              \\
  \hline
  \endhead

  \hline
  \endfoot

  \hline
  \caption{Phân chia các miền dữ liệu trong hệ thống} \label{tab:data-domains}                                                                                                                                                                           \\
  \endlastfoot

  1            & Xác thực \& Người dùng & \texttt{staff}, \texttt{sessions}                                                                                                                                                                              \\
  \hline
  2            & Quản lý ca \& Tiền mặt & \texttt{shifts}, \texttt{cash\_drops}                                                                                                                                                                          \\
  \hline
  3            & Nhân sự                & \texttt{shift\_templates}, \texttt{work\_schedules}, \texttt{timekeeping}, \texttt{leave\_requests}, \texttt{payroll\_periods}, \texttt{payroll\_records}                                                      \\
  \hline
  4            & Danh mục sản phẩm      & \texttt{categories}, \texttt{physical\_items}, \texttt{sales\_items}, \texttt{recipes}, \texttt{recipe\_items}, \texttt{combos}, \texttt{combo\_items}                                                         \\
  \hline
  5            & Quản lý kho            & \texttt{suppliers}, \texttt{receivings}, \texttt{receiving\_items}, \texttt{inventory\_batches}, \texttt{stock\_movements}, \texttt{conversions}, \texttt{inventory\_counts}, \texttt{inventory\_count\_items} \\
  \hline
  6            & Bán hàng               & \texttt{orders}, \texttt{order\_items}, \texttt{order\_item\_batches}, \texttt{payments}, \texttt{receipts}, \texttt{remakes}                                                                                  \\
  \hline
  7            & Tiêu hủy \& Kiểm soát  & \texttt{waste\_logs}                                                                                                                                                                                           \\
  \hline
  8            & Cấu hình \& Khuyến mại & \texttt{stores}, \texttt{payment\_method\_configs}, \texttt{printer\_configs}, \texttt{alerts}, \texttt{discount\_rules}, \texttt{discount\_rule\_items}                                                       \\
  \hline
\end{longtable}

\subsection{Mô tả chi tiết các bảng dữ liệu}

\subsubsection{Miền Xác thực và Người dùng}

\textbf{Bảng staff} - Lưu trữ thông tin nhân viên:

\begin{longtable}{|p{3.5cm}|p{3cm}|p{7cm}|}
  \hline
  \textbf{Thuộc tính} & \textbf{Kiểu dữ liệu} & \textbf{Mô tả}                        \\
  \hline
  \endfirsthead
  \endhead
  \endfoot
  \hline
  \caption{Cấu trúc bảng staff}                                                       \\
  \endlastfoot

  id                  & SERIAL                & Khóa chính, tự động tăng              \\
  \hline
  name                & VARCHAR(100)          & Họ tên nhân viên                      \\
  \hline
  email               & VARCHAR(150)          & Email đăng ký (dùng để khôi phục PIN) \\
  \hline
  pin\_hash           & VARCHAR(255)          & Mã PIN đã được băm                    \\
  \hline
  role                & VARCHAR(20)           & Vai trò: STAFF hoặc MANAGER           \\
  \hline
  store\_id           & INTEGER               & Khóa ngoại đến bảng stores            \\
  \hline
  hourly\_rate        & INTEGER               & Lương theo giờ (đơn vị: đồng)         \\
  \hline
  is\_active          & BOOLEAN               & Trạng thái hoạt động                  \\
  \hline
  created\_at         & TIMESTAMPTZ           & Thời điểm tạo                         \\
  \hline
  updated\_at         & TIMESTAMPTZ           & Thời điểm cập nhật cuối               \\
  \hline
\end{longtable}

\textbf{Bảng sessions} - Theo dõi phiên đăng nhập:

\begin{longtable}{|p{3.5cm}|p{3cm}|p{7cm}|}
  \hline
  \textbf{Thuộc tính} & \textbf{Kiểu dữ liệu} & \textbf{Mô tả}                                    \\
  \hline
  \endfirsthead
  \endhead
  \endfoot
  \hline
  \caption{Cấu trúc bảng sessions}                                                                \\
  \endlastfoot

  id                  & SERIAL                & Khóa chính                                        \\
  \hline
  staff\_id           & INTEGER               & Khóa ngoại đến staff                              \\
  \hline
  shift\_id           & INTEGER               & Khóa ngoại đến shifts                             \\
  \hline
  login\_at           & TIMESTAMPTZ           & Thời điểm đăng nhập                               \\
  \hline
  logout\_at          & TIMESTAMPTZ           & Thời điểm đăng xuất (được phép trống)             \\
  \hline
  logout\_reason      & VARCHAR(20)           & Lý do đăng xuất: MANUAL, AUTO\_LOCK, FAST\_SWITCH \\
  \hline
\end{longtable}

\subsubsection{Miền Quản lý ca và Tiền mặt}

\textbf{Bảng shifts} - Quản lý ca làm việc:

\begin{longtable}{|p{3.5cm}|p{3cm}|p{7cm}|}
  \hline
  \textbf{Thuộc tính} & \textbf{Kiểu dữ liệu} & \textbf{Mô tả}                      \\
  \hline
  \endfirsthead
  \endhead
  \endfoot
  \hline
  \caption{Cấu trúc bảng shifts}                                                    \\
  \endlastfoot

  id                  & SERIAL                & Khóa chính                          \\
  \hline
  staff\_id           & INTEGER               & Nhân viên mở ca                     \\
  \hline
  store\_id           & INTEGER               & Cửa hàng                            \\
  \hline
  start\_time         & TIMESTAMPTZ           & Thời điểm mở ca                     \\
  \hline
  end\_time           & TIMESTAMPTZ           & Thời điểm đóng ca                   \\
  \hline
  opening\_cash       & INTEGER               & Tiền mặt đầu ca                     \\
  \hline
  closing\_cash       & INTEGER               & Tiền mặt cuối ca (thực tế đếm)      \\
  \hline
  expected\_cash      & INTEGER               & Tiền mặt dự kiến (tính từ hệ thống) \\
  \hline
  cash\_variance      & INTEGER               & Chênh lệch (thực tế - dự kiến)      \\
  \hline
  variance\_reason    & VARCHAR(255)          & Lý do chênh lệch                    \\
  \hline
  status              & VARCHAR(20)           & Trạng thái: OPEN hoặc CLOSED        \\
  \hline
\end{longtable}

\subsubsection{Miền Nhân sự}

\textbf{Bảng timekeeping} - Chấm công nhân viên:

\begin{longtable}{|p{3.5cm}|p{3cm}|p{7cm}|}
  \hline
  \textbf{Thuộc tính} & \textbf{Kiểu dữ liệu} & \textbf{Mô tả}                            \\
  \hline
  \endfirsthead
  \endhead
  \endfoot
  \hline
  \caption{Cấu trúc bảng timekeeping}                                                     \\
  \endlastfoot

  id                  & SERIAL                & Khóa chính                                \\
  \hline
  staff\_id           & INTEGER               & Nhân viên chấm công                       \\
  \hline
  work\_date          & DATE                  & Ngày làm việc                             \\
  \hline
  clock\_in           & TIMESTAMPTZ           & Thời điểm vào làm                         \\
  \hline
  clock\_out          & TIMESTAMPTZ           & Thời điểm ra về                           \\
  \hline
  total\_hours        & DECIMAL(5,2)          & Tổng giờ làm                              \\
  \hline
  status              & VARCHAR(20)           & PENDING, CLOCKED\_IN, COMPLETED, ADJUSTED \\
  \hline
  adjusted\_by        & INTEGER               & Người điều chỉnh (nếu có)                 \\
  \hline
  adjustment\_reason  & VARCHAR(255)          & Lý do điều chỉnh                          \\
  \hline
\end{longtable}

\textbf{Bảng leave\_requests} - Đơn xin nghỉ phép:

\begin{longtable}{|p{3.5cm}|p{3cm}|p{7cm}|}
  \hline
  \textbf{Thuộc tính} & \textbf{Kiểu dữ liệu} & \textbf{Mô tả}                           \\
  \hline
  \endfirsthead
  \endhead
  \endfoot
  \hline
  \caption{Cấu trúc bảng leave\_requests}                                                \\
  \endlastfoot

  id                  & SERIAL                & Khóa chính                               \\
  \hline
  staff\_id           & INTEGER               & Nhân viên xin nghỉ                       \\
  \hline
  leave\_type         & VARCHAR(30)           & Loại nghỉ: ANNUAL, SICK, PERSONAL, OTHER \\
  \hline
  start\_date         & DATE                  & Ngày bắt đầu nghỉ                        \\
  \hline
  end\_date           & DATE                  & Ngày kết thúc nghỉ                       \\
  \hline
  reason              & TEXT                  & Lý do xin nghỉ                           \\
  \hline
  status              & VARCHAR(20)           & PENDING, APPROVED, REJECTED              \\
  \hline
  reviewed\_by        & INTEGER               & Quản lý duyệt đơn                        \\
  \hline
  reviewer\_notes     & VARCHAR(255)          & Ghi chú của quản lý                      \\
  \hline
\end{longtable}

\textbf{Bảng work\_schedules} - Đăng ký lịch làm việc:

\begin{longtable}{|p{3.5cm}|p{3cm}|p{7cm}|}
  \hline
  \textbf{Thuộc tính} & \textbf{Kiểu dữ liệu} & \textbf{Mô tả}                         \\
  \hline
  \endfirsthead
  \endhead
  \endfoot
  \hline
  \caption{Cấu trúc bảng work\_schedules}                                              \\
  \endlastfoot

  id                  & SERIAL                & Khóa chính                             \\
  \hline
  staff\_id           & INTEGER               & Nhân viên đăng ký                      \\
  \hline
  shift\_template\_id & INTEGER               & Ca làm việc mẫu                        \\
  \hline
  work\_date          & DATE                  & Ngày làm việc                          \\
  \hline
  status              & VARCHAR(20)           & PENDING, APPROVED, REJECTED, CANCELLED \\
  \hline
  approved\_by        & INTEGER               & Quản lý phê duyệt                      \\
  \hline
\end{longtable}

\subsubsection{Miền Danh mục sản phẩm}

\textbf{Bảng physical\_items} - Hàng hóa vật lý (SKU trong kho):

\begin{longtable}{|p{3.5cm}|p{3cm}|p{7cm}|}
  \hline
  \textbf{Thuộc tính} & \textbf{Kiểu dữ liệu} & \textbf{Mô tả}                       \\
  \hline
  \endfirsthead
  \endhead
  \endfoot
  \hline
  \caption{Cấu trúc bảng physical\_items}                                            \\
  \endlastfoot

  id                  & SERIAL                & Khóa chính                           \\
  \hline
  name                & VARCHAR(150)          & Tên sản phẩm vật lý                  \\
  \hline
  barcode             & VARCHAR(50)           & Mã vạch (tùy chọn)                   \\
  \hline
  storage\_unit       & VARCHAR(30)           & Đơn vị lưu kho (gói, thùng, vỉ)      \\
  \hline
  prepared\_unit      & VARCHAR(30)           & Đơn vị sau chế biến (miếng, quả)     \\
  \hline
  conversion\_ratio   & DECIMAL(10,4)         & Tỷ lệ chuyển đổi (1 gói = 10 miếng)  \\
  \hline
  expiry\_hours       & INTEGER               & Thời gian hết hạn sau chế biến (giờ) \\
  \hline
  reorder\_threshold  & INTEGER               & Ngưỡng cảnh báo bổ sung              \\
  \hline
\end{longtable}

\textbf{Bảng sales\_items} - Sản phẩm bán (SKU bán hàng):

\begin{longtable}{|p{3.5cm}|p{3cm}|p{7cm}|}
  \hline
  \textbf{Thuộc tính}   & \textbf{Kiểu dữ liệu} & \textbf{Mô tả}                               \\
  \hline
  \endfirsthead
  \endhead
  \endfoot
  \hline
  \caption{Cấu trúc bảng sales\_items}                                                         \\
  \endlastfoot

  id                    & SERIAL                & Khóa chính                                   \\
  \hline
  category\_id          & INTEGER               & Danh mục sản phẩm                            \\
  \hline
  physical\_item\_id    & INTEGER               & Liên kết đến hàng vật lý (nếu bán trực tiếp) \\
  \hline
  name                  & VARCHAR(150)          & Tên sản phẩm bán                             \\
  \hline
  price                 & INTEGER               & Giá bán (đồng)                               \\
  \hline
  near\_expiry\_price   & INTEGER               & Giá khi cận date                             \\
  \hline
  vat\_rate             & DECIMAL(5,2)          & Thuế suất VAT (0, 5, 8, 10\%)                \\
  \hline
  requires\_preparation & BOOLEAN               & Có cần chế biến không                        \\
  \hline
\end{longtable}

\textbf{Bảng recipes và recipe\_items} - Công thức chế biến:

Bảng \texttt{recipes} lưu thông tin công thức, liên kết 1-1 với \texttt{sales\_items} có \texttt{requires\_preparation = true}. Bảng \texttt{recipe\_items} lưu chi tiết nguyên liệu với số lượng cần thiết. Đây là cơ sở cho cơ chế \textbf{Backflushing} - tự động trừ kho nguyên liệu khi bán món chế biến.

\subsubsection{Miền Quản lý kho}

\textbf{Bảng inventory\_batches} - Quản lý lô hàng:

\begin{longtable}{|p{3.5cm}|p{3cm}|p{7cm}|}
  \hline
  \textbf{Thuộc tính} & \textbf{Kiểu dữ liệu} & \textbf{Mô tả}                       \\
  \hline
  \endfirsthead
  \endhead
  \endfoot
  \hline
  \caption{Cấu trúc bảng inventory\_batches}                                         \\
  \endlastfoot

  id                  & SERIAL                & Khóa chính                           \\
  \hline
  physical\_item\_id  & INTEGER               & Sản phẩm vật lý                      \\
  \hline
  receiving\_item\_id & INTEGER               & Nguồn từ phiếu nhập (nếu có)         \\
  \hline
  source\_batch\_id   & INTEGER               & Lô nguồn (nếu là kết quả chuyển đổi) \\
  \hline
  quantity\_initial   & INTEGER               & Số lượng ban đầu                     \\
  \hline
  quantity\_remaining & INTEGER               & Số lượng còn lại                     \\
  \hline
  created\_at         & TIMESTAMPTZ           & Thời điểm tạo lô                     \\
  \hline
  expires\_at         & TIMESTAMPTZ           & Thời điểm hết hạn                    \\
  \hline
  batch\_state        & VARCHAR(30)           & STORAGE hoặc PREPARED                \\
  \hline
\end{longtable}

Cơ chế FIFO được thực hiện bằng cách sắp xếp các lô theo \texttt{created\_at ASC} và trừ từ lô cũ nhất trước.

\textbf{Bảng stock\_movements} - Nhật ký biến động kho:

Đây là bảng trung tâm cho việc truy vết, ghi nhận mọi thay đổi tồn kho với các loại:
\begin{itemize}
  \item \texttt{RECEIVING}: Nhập hàng từ nhà cung cấp
  \item \texttt{SALE}: Bán hàng (trừ kho)
  \item \texttt{WASTE}: Tiêu hủy
  \item \texttt{EXPIRY}: Hết hạn tự động
  \item \texttt{CONVERSION\_IN/OUT}: Chuyển đổi trạng thái
  \item \texttt{ADJUSTMENT}: Điều chỉnh kiểm kê
  \item \texttt{VOID}: Hủy đơn hàng (hoàn kho)
\end{itemize}

\subsubsection{Miền Bán hàng}

\textbf{Bảng orders} - Đơn hàng:

\begin{longtable}{|p{3.5cm}|p{3cm}|p{7cm}|}
  \hline
  \textbf{Thuộc tính}   & \textbf{Kiểu dữ liệu} & \textbf{Mô tả}                                   \\
  \hline
  \endfirsthead
  \endhead
  \endfoot
  \hline
  \caption{Cấu trúc bảng orders}                                                                   \\
  \endlastfoot

  id                    & SERIAL                & Khóa chính                                       \\
  \hline
  staff\_id             & INTEGER               & Nhân viên tạo đơn                                \\
  \hline
  shift\_id             & INTEGER               & Ca làm việc                                      \\
  \hline
  order\_number         & VARCHAR(10)           & Số đơn hàng (4 chữ số, reset hàng ngày)          \\
  \hline
  subtotal              & INTEGER               & Tổng tiền hàng                                   \\
  \hline
  discount\_total       & INTEGER               & Tổng giảm giá                                    \\
  \hline
  vat\_total            & INTEGER               & Tổng VAT                                         \\
  \hline
  grand\_total          & INTEGER               & Tổng thanh toán                                  \\
  \hline
  status                & VARCHAR(30)           & DRAFT, PENDING\_PAYMENT, PAID, COMPLETED, VOIDED \\
  \hline
  remake\_of\_order\_id & INTEGER               & Đơn gốc (nếu là đơn làm lại)                     \\
  \hline
\end{longtable}

\textbf{Bảng order\_item\_batches} - Theo dõi FIFO khi bán:

Bảng này ghi nhận chi tiết việc trừ kho từ những lô nào cho mỗi dòng đơn hàng, đảm bảo khả năng truy vết và hỗ trợ hoàn kho chính xác khi hủy đơn.

\subsection{Biểu đồ quan hệ thực thể (ERD)}

\begin{landscape}
  \begin{figure}[p]
    \centering
    \includegraphics[width=\linewidth,height=\textheight,keepaspectratio]{assets/diagram-erd.png}
    \caption{Biểu đồ ERD tổng quan hệ thống CVS}
    \label{fig:erd-overview}
  \end{figure}
\end{landscape}

\subsection{Các ràng buộc toàn vẹn}

\subsubsection{Ràng buộc khóa chính và khóa ngoại}

Tất cả các bảng sử dụng khóa chính tự động tăng (\texttt{SERIAL}). Các quan hệ giữa các bảng được thực thi thông qua ràng buộc khóa ngoại với \texttt{ON DELETE RESTRICT} để ngăn chặn việc xóa dữ liệu đang được tham chiếu.

\subsubsection{Ràng buộc nghiệp vụ}

\begin{itemize}
  \item \textbf{FIFO bắt buộc:} Mọi thao tác trừ kho phải tuân thủ FIFO, không có ngoại lệ.
  \item \textbf{Không bán hàng hết hạn:} Hệ thống chặn việc bán từ các lô có \texttt{expires\_at < NOW()}.
  \item \textbf{Không sửa đơn hàng:} Đơn hàng sau khi thanh toán không thể sửa, chỉ có thể hủy và tạo lại.
  \item \textbf{Tiền lưu dạng số nguyên:} Tất cả giá trị tiền tệ lưu dưới dạng INTEGER (đơn vị: đồng) để tránh sai số làm tròn.
  \item \textbf{PIN được băm:} Mã PIN không lưu plaintext, chỉ lưu giá trị đã băm.
  \item \textbf{Soft delete cho audit:} Các bản ghi quan trọng (đơn hàng, phiếu tiêu hủy) không được xóa cứng, chỉ đánh dấu trạng thái.
\end{itemize}

\subsection{Quy ước đặt tên}

\begin{longtable}{|p{4cm}|p{4cm}|p{5cm}|}
  \hline
  \textbf{Đối tượng} & \textbf{Quy ước}                & \textbf{Ví dụ}                                  \\
  \hline
  \endfirsthead
  \endhead
  \endfoot
  \hline
  \caption{Quy ước đặt tên trong CSDL}                                                                   \\
  \endlastfoot

  Tên bảng           & snake\_case, số nhiều           & \texttt{order\_items}, \texttt{physical\_items} \\
  \hline
  Tên cột            & snake\_case                     & \texttt{created\_at}, \texttt{unit\_price}      \\
  \hline
  Khóa chính         & \texttt{id}                     & \texttt{id SERIAL PRIMARY KEY}                  \\
  \hline
  Khóa ngoại         & \texttt{\{bảng\}\_id}           & \texttt{staff\_id}, \texttt{order\_id}          \\
  \hline
  Chỉ mục            & \texttt{idx\_\{bảng\}\_\{cột\}} & \texttt{idx\_orders\_created\_at}               \\
  \hline
  Timestamp          & TIMESTAMPTZ                     & Lưu kèm timezone                                \\
  \hline
  Tiền tệ            & INTEGER                         & Đơn vị: đồng Việt Nam                           \\
  \hline
\end{longtable}

\section{Thiết kế giao diện người dùng}

\subsection{Chuẩn hoá cấu hình màn hình}

\subsubsection{Quy định về hiển thị}
\begin{itemize}
  \item \textbf{Độ phân giải:}
    \begin{itemize}
      \item Màn hình POS (Bán hàng): Tối ưu cho độ phân giải 1024x768 pixels (tỷ lệ 4:3) hoặc 1280x800 pixels, hỗ trợ thao tác cảm ứng với các nút bấm lớn.
      \item Màn hình Quản lý: Tối ưu cho độ phân giải 1366x768 pixels trở lên trên trình duyệt web máy tính.
    \end{itemize}
  \item \textbf{Thông báo:}
    \begin{itemize}
      \item \textbf{Thông báo lỗi/cảnh báo:} Hiển thị dạng Popup (Modal) ở chính giữa màn hình, làm mờ nền phía sau, yêu cầu người dùng xác nhận hoặc sửa lỗi trước khi tiếp tục.
      \item \textbf{Thông báo thành công:} Hiển thị dạng Toast message ở góc trên bên phải, tự động ẩn sau 3 giây.
    \end{itemize}
  \item \textbf{Định dạng dữ liệu:}
    \begin{itemize}
      \item \textbf{Số tiền:} Sử dụng dấu chấm (.) để phân cách hàng nghìn (Ví dụ: 100.000 đ). Đơn vị tiền tệ mặc định là VNĐ.
      \item \textbf{Ngày tháng:} Định dạng dd/MM/yyyy (Ví dụ: 30/01/2026).
      \item \textbf{Ký tự cho phép:} Chuỗi ký tự bao gồm chữ cái, chữ số, dấu phẩy, dấu chấm, dấu cách, dấu gạch dưới và gạch nối.
    \end{itemize}
\end{itemize}

\subsubsection{Cơ chế điều khiển}
\begin{itemize}
  \item \textbf{Kiểm tra dữ liệu đầu vào:}
    \begin{itemize}
      \item \textbf{Kiểm tra trường bắt buộc:} Kiểm tra các trường bắt buộc ngay khi người dùng rời khỏi ô nhập liệu (làm mờ) hoặc khi nhấn lưu.
      \item \textbf{Kiểm tra định dạng:} Kiểm tra định dạng đúng của Email, Số điện thoại, Mã vạch.
      \item \textbf{Logic nghiệp vụ:} Ví dụ: Số lượng tồn kho không được âm, Tiền khách đưa $\ge$ Tổng tiền thanh toán.
    \end{itemize}
  \item \textbf{Dịch chuyển màn hình:}
    \begin{itemize}
      \item Thiết kế theo hướng phẳng, hạn chế các khung chồng chéo.
      \item Các màn hình chức năng (Bán hàng, Kho, Nhân sự) được tách biệt rõ ràng thông qua thanh điều hướng.
      \item Các tác vụ phụ (như Hướng dẫn sử dụng, Chi tiết nhanh) có thể hiển thị dạng Popup đè lên màn hình chính, làm mờ màn hình dưới để tập trung sự chú ý.
    \end{itemize}
\end{itemize}

\subsubsection{Luồng màn hình hệ thống}

Trình tự xuất hiện các màn hình trong kịch bản sử dụng thông thường:

\begin{enumerate}
  \item \textbf{Khởi động:} Màn hình chờ $\rightarrow$ Màn hình Đăng nhập.
  \item \textbf{Trang chủ (Sau đăng nhập):}
    \begin{itemize}
      \item Với Nhân viên: Chuyển thẳng vào Màn hình POS (Bán hàng).
      \item Với Quản lý: Chuyển vào Dashboard (Bảng điều khiển thống kê).
    \end{itemize}
  \item \textbf{Các luồng chức năng chính:}
    \begin{itemize}
      \item \textbf{Bán hàng:} POS $\rightarrow$ Chọn món $\rightarrow$ Chỉnh sửa đơn (Topping/Ghi chú) $\rightarrow$ Thanh toán (Tiền mặt/QR) $\rightarrow$ In hóa đơn $\rightarrow$ Kết thúc.
      \item \textbf{Quản lý kho:} Danh sách nhập/xuất $\rightarrow$ Tạo phiếu nhập/xuất $\rightarrow$ Chọn hàng hóa $\rightarrow$ Xác nhận $\rightarrow$ Cập nhật tồn kho.
      \item \textbf{Nhân sự:} Lịch làm việc $\rightarrow$ Đăng ký ca $\rightarrow$ Xin nghỉ phép $\rightarrow$ Chấm công.
    \end{itemize}
  \item \textbf{Màn hình lỗi (Error):} Khi hệ thống gặp sự cố, một thông điệp rõ ràng sẽ hiện lên thông báo vấn đề và hướng dẫn cách khắc phục (thử lại hoặc liên hệ kỹ thuật).
\end{enumerate}

\subsection{Các ảnh màn hình}

% =============================================
% NHÓM 1: ĐĂNG NHẬP & NHÂN VIÊN BÁN HÀNG
% =============================================

\begin{figure}[H]
  \centering
  \includegraphics[width=0.9\textwidth]{assets/anh_man_hinh/dang_nhap.png}
  \caption{Màn hình đăng nhập}
  \label{fig:dang_nhap}
\end{figure}

\begin{figure}[H]
  \centering
  \includegraphics[width=0.9\textwidth]{assets/anh_man_hinh/nhan_vien_ban_hang.png}
  \caption{Màn hình nhân viên bán hàng}
  \label{fig:nhan_vien_ban_hang}
\end{figure}

\begin{figure}[H]
  \centering
  \includegraphics[width=0.9\textwidth]{assets/anh_man_hinh/lich_lam_viec.png}
  \caption{Màn hình nhân viên bán hàng - Lịch làm việc}
  \label{fig:lich_lam_viec}
\end{figure}

\begin{figure}[H]
  \centering
  \includegraphics[width=0.9\textwidth]{assets/anh_man_hinh/dang_ky_ca_lam.png}
  \caption{Lịch làm việc - Đăng ký ca làm}
  \label{fig:dang_ky_ca_lam}
\end{figure}

\begin{figure}[H]
  \centering
  \includegraphics[width=0.9\textwidth]{assets/anh_man_hinh/xin_nghi.png}
  \caption{Đơn xin nghỉ phép}
  \label{fig:xin_nghi}
\end{figure}

% =============================================
% NHÓM 2: POS, THANH TOÁN & BẾP
% =============================================

\begin{figure}[H]
  \centering
  \includegraphics[width=0.9\textwidth]{assets/anh_man_hinh/pos.png}
  \caption{Màn hình POS}
  \label{fig:pos}
\end{figure}

\begin{figure}[H]
  \centering
  \includegraphics[width=0.9\textwidth]{assets/anh_man_hinh/che_bien.png}
  \caption{Màn hình POS - Quản lý thực phẩm chế biến}
  \label{fig:che_bien}
\end{figure}

\begin{figure}[H]
  \centering
  \includegraphics[width=0.9\textwidth]{assets/anh_man_hinh/QR.png}
  \caption{Màn hình thanh toán của khách hàng (Quét QR)}
  \label{fig:qr}
\end{figure}

\begin{figure}[H]
  \centering
  \includegraphics[width=0.9\textwidth]{assets/anh_man_hinh/hoan_tien_lam_lai.png}
  \caption{Màn hình Hoàn tiền / Làm lại món}
  \label{fig:hoan_tien_lam_lai}
\end{figure}

\begin{figure}[H]
  \centering
  \includegraphics[width=0.9\textwidth]{assets/anh_man_hinh/quan_ly_nhap_xuat.png}
  \caption{Quản lý nhập xuất kho}
  \label{fig:nhap_xuat}
\end{figure}

\begin{figure}[H]
  \centering
  \includegraphics[width=0.9\textwidth]{assets/anh_man_hinh/popup_ket_thuc_ca.png}
  \caption{Popup kết thúc ca làm việc}
  \label{fig:popup_ket_thuc_ca}
\end{figure}

% =============================================
% NHÓM 3: QUẢN LÝ (DASHBOARD & THIẾT LẬP)
% =============================================

\begin{figure}[H]
  \centering
  \includegraphics[width=0.9\textwidth]{assets/anh_man_hinh/man_hinh_quan_ly.png}
  \caption{Giao diện quản lý chính}
  \label{fig:man_hinh_quan_ly}
\end{figure}

\begin{figure}[H]
  \centering
  \includegraphics[width=0.9\textwidth]{assets/anh_man_hinh/dashboard_tong_quan.png}
  \caption{Hệ thống quản lý - Dashboard tổng quan}
  \label{fig:dashboard}
\end{figure}

\begin{figure}[H]
  \centering
  \includegraphics[width=0.9\textwidth]{assets/anh_man_hinh/cong_thuc_che_bien.png}
  \caption{Quản lý công thức chế biến}
  \label{fig:cong_thuc}
\end{figure}

\begin{figure}[H]
  \centering
  \includegraphics[width=0.9\textwidth]{assets/anh_man_hinh/quan_ly_combo.png}
  \caption{Quản lý Combo}
  \label{fig:combo}
\end{figure}

\begin{figure}[H]
  \centering
  \includegraphics[width=0.9\textwidth]{assets/anh_man_hinh/quan_ly_khuyen_mai.png}
  \caption{Quản lý khuyến mại}
  \label{fig:khuyen_mai}
\end{figure}

% =============================================
% NHÓM 4: KHO & DANH MỤC HÀNG HÓA
% =============================================

\begin{figure}[H]
  \centering
  \includegraphics[width=0.9\textwidth]{assets/anh_man_hinh/quan_ly_danh_muc_nhap.png}
  \caption{Quản lý danh mục hàng - Hàng nhập}
  \label{fig:dm_nhap}
\end{figure}

\begin{figure}[H]
  \centering
  \includegraphics[width=0.9\textwidth]{assets/anh_man_hinh/quan_ly_danh_muc_ban.png}
  \caption{Quản lý danh mục hàng - Hàng bán}
  \label{fig:dm_ban}
\end{figure}


\begin{figure}[H]
  \centering
  \includegraphics[width=0.9\textwidth]{assets/anh_man_hinh/quan_ly_nhap_xuat_kho.png}
  \caption{Chi tiết danh sách nhập xuất kho}
  \label{fig:nhap_xuat_kho}
\end{figure}

\begin{figure}[H]
  \centering
  \includegraphics[width=0.9\textwidth]{assets/anh_man_hinh/phieu_nhap_kho.png}
  \caption{Phiếu nhập kho}
  \label{fig:phieu_nhap_kho}
\end{figure}

\begin{figure}[H]
  \centering
  \includegraphics[width=0.9\textwidth]{assets/anh_man_hinh/phieu_huy_hang.png}
  \caption{Phiếu hủy hàng}
  \label{fig:phieu_huy_hang}
\end{figure}

\begin{figure}[H]
  \centering
  \includegraphics[width=0.9\textwidth]{assets/anh_man_hinh/phieu_kiem_ke.png}
  \caption{Phiếu kiểm kê kho}
  \label{fig:phieu_kiem_ke}
\end{figure}

% =============================================
% NHÓM 5: NHÂN SỰ & HỆ THỐNG
% =============================================

\begin{figure}[H]
  \centering
  \includegraphics[width=0.9\textwidth]{assets/anh_man_hinh/danh_sach_nhan_vien.png}
  \caption{Quản lý danh sách nhân viên}
  \label{fig:ql_nhan_vien}
\end{figure}

\begin{figure}[H]
  \centering
  \includegraphics[width=0.9\textwidth]{assets/anh_man_hinh/phan_ca_lam_viec.png}
  \caption{Phân ca làm việc}
  \label{fig:phan_ca}
\end{figure}

\begin{figure}[H]
  \centering
  \includegraphics[width=0.9\textwidth]{assets/anh_man_hinh/quan_ly_tai_khoan.png}
  \caption{Quản lý tài khoản}
  \label{fig:ql_tai_khoan}
\end{figure}

\begin{figure}[H]
  \centering
  \includegraphics[width=0.9\textwidth]{assets/anh_man_hinh/cau_hinh_he_thong.png}
  \caption{Cấu hình hệ thống}
  \label{fig:cau_hinh}
\end{figure}

\begin{figure}[H]
  \centering
  \includegraphics[width=0.9\textwidth]{assets/anh_man_hinh/bao_cao_thong_ke.png}
  \caption{Báo cáo thống kê}
  \label{fig:bao_cao}
\end{figure}

\subsection{Sơ đồ chuyển màn hình}
\begin{figure}[H]
  \centering
  \includegraphics[width=0.9\textwidth]{assets/anh_man_hinh/chuyen_man_hinh.png}
  \caption{Sơ đồ chuyển màn hình}
  \label{fig:chuyen_man_hinh}
\end{figure}
% Chèn sơ đồ nếu có
% \includegraphics...

\newpage
%\pagestyle{fancy} % Áp dụng header và footer
% \label{chapter:SolutionAndContribution}
% !TEX root = ../main.tex
\chapter{CHƯƠNG TRÌNH MINH HOẠ}


\newpage
%\pagestyle{fancy} % Áp dụng header và footer
% \label{chapter:conclusion}
% !TEX root = ../main.tex
\chapter{NGUYÊN LÝ THIẾT KẾ} %Kết luận và hướng phát triển}


\newpage
% !TEX root = ../main.tex
\chapter{KẾT LUẬN VÀ HƯỚNG PHÁT TRIỂN}
%\pagestyle{fancy} % Áp dụng header và footer

% ===================================================

\setcounter{biburllcpenalty}{7000}
\setcounter{biburlucpenalty}{8000}
\printbibliography[heading=bibintoc, title={TÀI LIỆU THAM KHẢO}] % Tùy chọn [heading=bibintoc] để thêm mục này vào Mục lục

% \appendixpage % (*) Lệnh này tạo ra một trang bìa riêng cho phần Phụ lục.
% \appendices % (*) Lệnh này báo cho LaTeX biết rằng tất cả các \chapter sau đó sẽ được coi là Phụ lục.
% \addappheadtotoc % (*) Lệnh này thêm một dòng tiêu đề chung cho phần Phụ lục vào Bảng Mục lục, giúp phân tách rõ ràng phần nội dung chính và phần phụ lục.

\titleformat{\chapter}[hang]{\centering\bfseries}{ \thechapter.\ }{0pt}{}[]
\titlespacing*{\chapter}{0pt}{-20pt}{20pt}

\titlecontents{chapter}
[0.0cm]             % left margin
{\bfseries\vspace{0.3cm}}                  % above code
{{\bfseries{\scshape} \thecontentslabel.\ }} % numbered format
{}         % unnumbered format
{\titlerule*[0.3pc]{.}\contentspage}         % filler-page-format, e.g dots

% \chapter*{PHỤ LỤC} %Kết luận và hướng phát triển}

%  \chapter{HƯỚNG DẪN VIẾT ĐỒ ÁN TỐT NGHIỆP}
%  \input{chapters/Phu_luc_A}
%
%  \newpage
%
%  \chapter{ĐẶC TẢ USE CASE}
%  \input{chapters/Phu_luc_B}

\end{document}
